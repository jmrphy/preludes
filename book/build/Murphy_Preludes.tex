\documentclass[a4paper,12pt,margin=.5in]{article}

\usepackage{xltxtra}
\usepackage{setspace}
\usepackage{endnotes}


\def\tightlist{}


\usepackage{url}
\usepackage[usenames,dvipsnames]{xcolor}
\usepackage[xetex]{hyperref}
\hypersetup{%
	backref=true,
    colorlinks=true,
    linkcolor=MidnightBlue,
    anchorcolor=MidnightBlue,
    citecolor=MidnightBlue,
    urlcolor=MidnightBlue,
    pdftitle    = {Preludes to a Science of Insurrection},%
    pdfkeywords = {revolution,insurrection,institutions,political science},
    pdfauthor   = {},%
}


\title{Preludes to a Science of Insurrection}

\author{\small }

%\author{}

%\author{}

\date{}

\begin{document}  
\setkeys{Gin}{width=1\textwidth} 	
\setromanfont[Mapping=tex-text,Numbers=OldStyle]{Georgia} 
\setsansfont[Mapping=tex-text]{Georgia} 
\setmonofont[Mapping=tex-text,Scale=0.8]{Georgia}
%\chapterstyle{article-4} 
%\pagestyle{kjh}

%\published{. Incomplete Draft. Please do not cite without permission.}

\maketitle

\footnotetext[1]{Last update: \today. This is a draft, please do not circulate. Please send any comments or questions to \href{mailto:j.murphy@soton.ac.uk}{j.murphy@soton.ac.uk}. Current word count: ~39,000.}

\newpage

{
\hypersetup{linkcolor=MidnightBlue}
\setcounter{tocdepth}{3}
\setcounter{secnumdepth}{1}
\doublespacing
\tableofcontents
\onehalfspacing
}

\clearpage
\onehalfspacing

\section{Preludes}\label{preludes}

\begin{enumerate}
\def\labelenumi{\arabic{enumi}.}
\item
  The aphorism may have to become a genre of special importance to any
  sincerely radical intellectual, for the simple reason that anyone
  committed to writing down dangerous truths can only do so by snatching
  them quickly and irregularly from one's ever fleeting and dwindling
  free time. Obviously the blog post as a genre testifies to this, but
  the mundane group psychology and attention economies of the web impose
  various conservative pressures on those who would try to think through
  blogging. When so much of our thought and feeling is pre-channeled
  into status quo institutions (thoughts and feelings which we subtly
  but irredeemably falsify as the price we pay to have them taken
  seriously by the relatively more powerful), one of the only ways to
  maintain an honest, living, dangerous intellectual project is to
  constantly overflow the cup of one's life---already filled to the brim
  by alienated work---and accumulate every little splash on any old rag
  at hand. Despite the unpolished, incomprehensible, inane, and even
  incorrect elements which will be inevitable, one's strung-together,
  splash-soaked rags will constitute a work of intellectual quality in
  crucial ways superior to most systematic works. While systematic works
  may be nicely filled cups, they are almost always funded by quotidian
  status quo hypocrisies which are required for their production but
  erased from the product; aphorisms are crystals formed on those
  splashes of honesty and truth squeezed out by such hypocrisies,
  preludes to something larger and more dangerous. In my life so far I
  have probably spent too much time making my silly little splashes and
  not enough tending the cup, but whether that has been my error, my
  crime, or my brilliance, only time will tell.
\item
  To seek truth passionately does not imply one is uniquely capable of
  attaining it; so often such a passion is interpreted as arrogance. On
  the contrary, it indicates a uniquely energetic and honest form of
  modesty.
\item
  I have no vision for the future of society but this does not make my
  political philosophy any less of a world-historical, revolutionary
  rupture certain to change everything---for clearly it is!---it only
  means I am not a megalomaniac.
\item
  Anyone who would wish to be an artist or intellectual is responsible
  for generating energy in others, an energy which is essentially
  ethical. As status quo institutions continue to strangle anyone
  childish enough to pursue such a task---as the institutions become
  exponentially more efficient since the information revolution---we
  observe nothing less than a society immunizing itself against the very
  possibility of becoming ethical. That such a remarkable process has
  also become so invisible, not even palpable by some of the most
  sensitive souls, only reflects how rapidly and completely several core
  human faculties have been destroyed in the course of about forty
  years.
\item
  State, party, faction, cadre, ``safe space''---a lineage.
\item
  There are still people on the radical left who do the selling
  newspapers thing. Just saying.
\item
  Being who one truly is: always the most radical thing one can do,
  individually. At first this sounds trite, until one recalls that it
  invokes authenticity, a non-starter for many.
\item
  Most of the benefits supposedly provided by status quo institutions
  fail to deliver their advertised satisfactions, but everyone goes on
  pretending to receive them, lest one be judged as incapable of
  happiness. The more success one has within the institutions, the more
  intense is this phenomenon. To have a comparatively decent perch in
  the world yet find almost all of its offerings unsatisfying appears as
  an absurd kind of ungratefulness. But this could not be more wrong:
  for people to traverse supposedly desirable stations and publicly out
  them for their emptiness is a first-class insurrectionary tactic. Such
  passings-through, which are also passings-over, concretely increase
  the truth and freedom of those touched while also leaving the enemy
  terrain even less habitable for all who will traverse it in the
  future.
\item
  In the bourgeois professions it is hard to complain about work with a
  straight face, so the dissatisfactions pile up repressed behind smiles
  until the face grows weak and dull, for the face is no longer hiding a
  true self but rather revealing what one's self has truly become.
\item
  Every generation eventually begins to suspect all is going to hell in
  a hand basket. The ignorant and the highly educated tend to agree in
  dismissing such pronouncements as unreliable, cyclical conservatism;
  because they have recurred like clockwork since time immemorial, such
  judgments are no longer seen as possibly indicating any real, new,
  objective degeneration of culture but only relative differences of
  perspective across generations. For this reason, neither the naive nor
  the ``critical'' can take seriously what now seems the most
  objectively likely possibility. Even after controlling for cyclical
  generational dynamics in perception, with every passing generation the
  world \emph{does} fall a little closer to hell, in a hand basket.
  Global warming indeed.
\item
  Whenever I am with a group, I always ask myself: What degree of this
  is theater? If a lot, I stay. If very little, I stay. Otherwise, I run
  for the door.
\item
  I have made something, I have priced it at negative infinity and I am
  smacking it out of your hand while saying ``you break it, you buy
  it.''
\item
  One must always keep the thread which runs from the beginning of one's
  life to the end. At any time, it can always be re-spun from whole
  cloth, but if one does not have---or cannot create---some thread
  running from the beginning to the end, this is the definition of being
  lost.
\item
  Hatred will never be a liberating force without the generosity to love
  one's enemies, for it is only through generosity that one can desire
  to destroy the evil in someone \emph{for their own good.} One reason
  why neither theoretical critique nor radical political projects do
  very much today is that people are far too happy to merely hate their
  enemies.
\item
  Listening to someone whose ideas I cannot take seriously, I go into a
  daze which is kind and attentive but also akin to sleep; I become
  surrounded by a haze of fog, composed of love, contempt,
  self-righteousness, and nothingness---in what proportions I can never
  determine.
\item
  Happiness derived from sunlight falsely inflates optimism; grey
  weather is depressing but grounding and for that reason more truthful.
\item
  Anyone who thinks social justice is more important than seeking the
  truth does not understand social justice and should be given what they
  are asking for: to be left alone.
\item
  People do not get more conservative after they fall in love, they just
  gain something worth conserving. The old puzzle about how young
  revolutionaries so easily settle into politically tranquil family
  lives is not a puzzle: family is typically the most intensely
  communist organization possible for most people in any society as
  atomized and pacified as ours. It is illusory to imagine that ``family
  values'' are conservative and the revolutionary path requires infinite
  openness to unconstrained sexual exploration and flexibility of
  commitments, etc. The family is just one tactic among others and in
  many cases one family contains more revolutionary potential than most
  radical subcultures.
\item
  The more carefully one observes the police, the more one's hatred
  turns to pity and pity turns to contempt. Disinterested observation
  implies a privileged distance for sure; nonetheless, it is only when
  contempt then turns to laughter and forgetting that hatred of the
  police has completed itself. It is much better to spread the privilege
  necessary for this process to permeate the body politic than valorize
  as political culture a stillborn hatred little more than love turned
  bitter.
\item
  There have been times I have fallen in love but then realized I was
  only inventing it precisely to avoid the disappointing fact that I was
  not falling in love. True love is no different, except one cares
  enough about the other to do this for eternity. In this, one finds a
  precious, secret joy which becomes a new reason to remain---perhaps,
  finally, a true one.
\item
  Modernity, democracy, capitalism: elites pretend to relinquish power
  to individuals but really only relinquish direct control---in favor of
  institutional manipulations which only modulate aggregate
  distributions of behavior, in exchange for a net increase in power,
  which comes as a cut of the profits derived from the new, more
  efficient equilibrium.
\item
  One can only have good character to the degree one is a character, in
  other words, an actor, a fake.
\item
  Most people need money, not radical politics. Most ``radical''
  politics need more people, not money.
\item
  In Barclay Shields I got out of my system in about one year what many
  other young adults are now inheriting as a norm overtaking the basic
  ethical substance of their lives, namely, that saying and doing things
  on the internet can feel good.
\item
  I much prefer religious zealotry to vulgar atheism or so-called
  progressive religion, for only true zealotry can transcend the falsity
  of instrumental reason. In no way is this invalidated by the promise
  of other-worldly rewards, which mean nothing and only serve as a
  bridge to allow the great mass of fallen mortals to even fathom the
  otherwise incommensurable mode of life which is true religion. It
  always appears to others as if the believer \emph{must} be driven by
  the instrumental purpose of obtaining some reward---but this is only
  because fallen mortals simply cannot access an experience of life
  beyond instrumental reason; that is what makes them fallen. The mortal
  thinks the believer is submitting to a primitive, traditional
  conception of reality in order to gain the benefits of a spiritual
  crutch in this world and fantastic rewards in the next, but the exact
  opposite is true: the faithful refuse to submit to the tyranny of
  empirical reality---the most insidious order of dishonest anasthesias
  and false prizes ever known---in favor of a \emph{true life now},
  despite all indications against its possibility and with no guarantees
  of any rewards whatsoever.
\item
  Multiple people will write with indignation ``Why is nobody talking
  about \emph{X}?''---pretending not to realize they are talking about
  \emph{X} as soon as humanly possible after \emph{X} happened. Why
  don't they just write something about it themselves and fill the gap
  they pretend to be concerned about? In most cases it is because they
  don't have anything new, interesting, or valuable to say, i.e.~they
  are not talking about it for the same reason as those maligned for
  their silence, namely, nobody has anything worth saying because
  typically it is exceedingly hard and rare to find something worth
  saying. A thing is worth saying to the degree it feeds into some
  substantively active, living process of which it is a part. In an
  atomized and pacified society such as ours, writing and speaking are
  rarely if ever part of a truly living process. Therefore, it is very
  poorly appreciated the degree to which ``critical'' media
  personalities and even some of our most intelligent friends on social
  media already long ago quit the game of trying to say anything
  uniquely valuable. The typical ``hot take'' does not even try to
  convey new or unique information to its readers beyond the information
  they are already expected to have, i.e.~such odd artifacts are
  sometimes literally, perfectly uninformative in the technical sense of
  the term. Well then, what are these people doing, from what land comes
  this fever in which one so loudly has nothing to say? Such heavy souls
  are selling affects for pocket change, whether symbolic or
  material---selling the very performance of their life---to then be
  pawned off by those others who possess even less ability to produce
  their own original affects, let alone new information. This may be all
  that is new in ``new media.''
\item
  When all of one's ``free time'' is robbed, the entire social game of
  liberalism ends and becomes a race to the death. It implies certain
  parties to the social game of liberalism have been vanquished not only
  in the limited warfare of economic competition but altogether as human
  lives. Continuing to play such a game as if it is not already over
  brings a little bit of certain death every passing day. In contrast,
  each day one instead chooses to live is no longer a ``day off'' but a
  re-entry into the historical mortal combat from which liberal
  democracy was only a short-lived and illusory reprieve. This prospect
  is vertiginous until it is recalled one is already standing on the
  gallows. To risk death in a mad dash for freedom is certainly no more
  frightening; it only seems more frightening to those who do not think
  honestly about where they already stand. For it is only by choosing
  life that one can win a race to the death.
\item
  I try to live so as to optimize the insurrectionary energy that would
  follow if I were unjustly killed, imprisoned, or disappeared. Another
  way to put this: I have been lucky enough to learn first hand that
  revolutionary politics is probably no more and no less than building
  true relationships and fighting to keep them. Incidentally, this is
  also one reason why the decline of organized religion has been a
  world-historical catastrophe, for whether we like it or not religion
  is the only basis humans have ever had, at least so far, for forging
  the kinds of relationships which make resistance to evil
  possible---relationships which dare to exist on the level of life or
  death.
\end{enumerate}

\section{Essays}\label{essays}

\subsection{The meaning of the revolutionary
position}\label{the-meaning-of-the-revolutionary-position}

If I insist on the revolutionary position, it is not to insist on the
dichotomy between revolution and reform. Most of us today will agree
with Gorz that there exists a class of revolutionary reforms, at which
point the relevant distinction becomes the distinction between
revolutionary reforms and reformist reforms. Today, Nancy Fraser
suggests the critical distinction is between ``system-conforming''
changes and ``system-transforming'' changes, but it seems to me that the
long-standing theoretical and practical difficulty remains the same:
which types of projects (individual or collective) effectively oppose
capitalism and push society toward justice, and which types of projects
(whether through mystification, co-optation, or defeat) merely improve
capitalism for some at the price of renouncing the system-level
opposition which would be the maximally true, coherent, and just
position.

Yet, to my mind, this is the essence of the revolutionary position: To
believe that the organisation of the world's institutions are unjust, to
see empirically that a key feature of these institutions is precisely
that they offer particular groups small gains in return for their
renunciation of system-level opposition, to therefore locate this
precise mechanism as the essential and perhaps only mechanism which is
able to maintain such massive worldwide system-level injustice, and
finally to assume the theoretical and practical position \emph{to never
renounce system-level opposition in exchange for any particular gain
less than the absolute system-level transformations which are required
for justice}, no matter how \emph{relatively} transformative such gains
might be.\footnote{This phrasing is purposely agnostic about what
  exactly constitutes justice or what any ultimate institutional
  configuration should look like (or how this would be determined). This
  is because, for the moment, I am trying to sketch what is essential
  and specific about the revolutionary position as inclusively as
  possible with respect to any particular vision of political justice.
  Thus, the only essential premises with which one has to agree here
  are: 1) that there currently exist system-level injustices in the
  arrangement of institutions, and 2) that we can at least in principle
  admit the possibility of a globally just arrangement of institutions.
  One does not even have to agree that capitalism is the name of the
  currently unjust institutions, to see how a commitment to system-level
  injustice necessarily implicates one at least in the \emph{question}
  of revolution.}

Because of the almost primordial or, in any event, perennial quality of
this tension and its unavoidable need for resolution in any
theoretically defensible political project, I see no way that any
political theory today can innocently elide the question of revolution.
I do not say that any political theory today must be explicitly
revolutionary in any specific sense. I say only that distinctions
between ``system-conforming'' and ``system-transforming'' beg the
crucial question which will always arise for those who agree to pursue
system-transforming collective action: when the state, the market,
and/or the thousands of institutions such as the university (defined by
irrevocable cognitive and material allegiances to the state and market)
offer us a particular ``transformation'' on condition that we demobilise
just enough to not threaten the equilibrium of the institutional
arrangement as such, \emph{should we accept that transformation or not?}

What are the conditions under which it is justified to demobilise
system-transformative activity in exchange for some political victory
which improves the world but falls short of absolute justice? This is
the perennial dilemma with which all system-transforming political
projects constantly struggle. The revolutionary position is nothing
other than answering this question with ``never.'' No matter how naively
romantic the revolutionary position rings to our contemporary ears, it's
naiveté and romanticism are only a function of the contingent dominance
of capitalism, our aversion to the apparent romanticism of the
revolutionary position is merely the cognitive inheritance of
generations of mystfied capitulation. Perhaps the meaning of the
revolutionary position is indeed nothing more than an integral naiveté,
but on the wager that real integrity to a truth is exactly the most
emancipatory political force in the world. Perhaps the most dangerous
romanticism existing today is the notion that humans have suddenly been
absolved of having to decide whether they will negotiate with oppressive
institutions or overthrow them.

\subsection{Ethically-biased technological
change}\label{ethically-biased-technological-change}

There is a widespread consensus in the economics literature that the
rise of information technology over the past several decades has been a
``skill-biased technological change.'' A skill-biased technological
change increases the productivity of skilled workers more than it
improves the productivity of unskilled workers, therefore increasing the
demand for skilled workers relative to unskilled workers. This increased
demand for skilled workers, ceteris paribus,
\href{http://www.dictionaryofeconomics.com/article?id=pde2008_S000493}{increases
their income relative to unskilled workers}. It is widely agreed that
this is a key explanation for at least some of the rapid increase in
income inequality since the 1980s.\footnote{See Card, David, and John E
  DiNardo. 2002. ``Skill Biased Technological Change and Rising Wage
  Inequality: Some Problems and Puzzles'' and Guvenen, Fatih, and
  Burhanettin Kuruscu. 2010. ``A Quantitative Analysis of the Evolution
  of the U.S. Wage Distribution, 1970--2000.'' NBER Macroeconomics
  Annual 24(1): 227--76.} One key observation consistent with this
theory is that it is only after the appearance of micro-computers that
income inequality begins to rise in the early 1980s.\footnote{Katz,
  Lawrence F. 2002. ``Technological Change, Computerization, and the
  Wage Structure.'' In Understanding the Digital Economy Data, Tools,
  and Research, Cambridge: MIT Press.} A second key observation is that
more educated workers are more likely to use a computer on the
job\footnote{Autor, David H, Lawrence F Katz, and Alan B Krueger. 1997.
  ``Computing Inequality: Have Computers Changed the Labor Market?''}.

Despite this growing interest in the economic implications of
technological change, social scientists have failed to take seriously a
surprisingly similar line of thought advanced most notably in
twentieth-century German philosophy. Heidegger and, later, the Frankfurt
School theorists developed critical theories of technology which argued,
in different ways, that technological changes shape ethical attitudes
toward the world. In particular, a specific line of thought revolves
around the relationship between technological development and
instrumental rationality, or in other words formal analytical
rationality which renounces consideration of substantive goals and
concerns itself only with calculating the optimal means toward given
ends.

For Heidegger, from a phenomenological perspective, technology
represents a ``clearing'' of Being which generates a relation to the
world in which nature and other humans are merely ``standing-reserve''
waiting to be exploited as means to whatever ends. For Adorno,
Horkheimer, and especially Marcuse, modern technological advancements
were ``organizing and perpetuating (or changing) social relationships, a
manifestation of prevalent thought and behavior patterns, an instrument
for control and domination.''\footnote{Marcuse, H. 1941. ``Some Social
  Implications of Modern Technology.'' Studies in Philosophy and Social
  Science, Vol. IX.}

While positivist social scientists have long perceived these types of
claims from continential social theory to be hopelessly speculative and
empirically intractable, this is puzzling given that the core claim
emerging from this particular current in the critical theory of
technology is relatively straightforward: technological development
increases the prevalence of instrumentalism among other possible ethical
attitudes toward the world. Moreover, the theory of skill-biased
technological change is a powerful heuristic for modeling this argument
given that it is a well-established model of how technology can generate
unequal distributions of power without any reference to Marxian bugaboos
which have likely caused more confusion than clarity by now.\footnote{I
  have in mind such perennial bad habits as lazily but self-certainly
  referring to Capital as a conspiratorial agent, an argument which has
  never boded very well for Marxism.}

However, it is perhaps more fair to observe that critical theorists
indeed sometimes lack what contemporary social scientists would call
``mechanisms,'' or specifiable and testable triggers which effectively
do the work of some variable \emph{X} supposedly causing some variable
\emph{Y}. While this particular critique of technological development is
not delivered with any particular ``mechanism'' which would be
intelligible to many empirical social scientists working today, critical
theory does supply many useful hints in this direction. Moreover, the
basic argument is surprisingly consistent with well-known economic
models, however disparate might be the vocabularies, thus permitting the
elaboration of a critical theory of technological change empirically
testable and consistent with a rational-choice framework.

To better specify the mechanism whereby technological change may
plausibly generate a greater prevalence of instrumental ethics, I
propose a theory of \emph{ethically-biased} technological change. I
argue that certain technological changes can increase the income paid to
certain ethical orientations relative to others. As in the theory of
skill-biased technological change, certain technological changes
increase the productivity of certain ethical orientations relative to
others, increasing their relative demand, and therefore increasing their
relative income. This increase in relative income represents a premium
paid to those with the ethical resources required by the new technology,
just as skill-biased technological change generates a premium paid to
those with the technical resources required by the new
technology.\footnote{Indeed, the analogy between skill-biased and
  ethically-biased technological change is all the more apt if we, with
  Foucault, conceptualize the practice of ethos as itself a kind of
  technical game; if ethical resources are themselves understood as
  technologies of the self. Key examples in this context are those most
  bourgeois of ethical resources: the various techniques of repressing
  ethical compunctions and moral anxieties, and the various techniques
  of guilt management consequently required by this ethos. In this
  register, ethically-biased technological change would be essentially
  identical to skill-biased technological with the exception that the
  ``skills'' in question are the skills of self-management rather than
  economic production. The reason we must avoid this reductionism at the
  outset is that it would beg the most important question, namely, the
  question regarding \emph{how} exactly did the contemporary human
  become, quite peculiar in the long-run of world history, so
  instrumentalist that the only activities remaining in contemporary
  life are exploiting and suffering exploitation?}

More specifically, a technological change which increases productivity
but transgresses \emph{any particular ethical value} will increase the
productivity of individuals with the weakest commitment to that ethical
value relative to those more firmly committed to that ethical value,
thus increasing demand for them relative to the more ethical
individuals, and thereby increasing their relative income. Ethically
transgressive technological innovations increase the productivity of the
least ethical because ethical compunctions are effectively a
psychological tax on a worker. It is hard to take a job, and then
demoralizing to work it, to the degree one feels ethical compunction for
doing so. While an employer would have to pay very high wages to attract
highly ethical workers to an ethically transgressive job, the wages they
have to pay to attract relatively unethical workers will tend to be
lower. The firmer one's ethical commitment, the greater is the tax one
pays to work an ethically transgressive job, the less worthwhile the
wage, and the less likely is one to take the job; the weaker one's
ethical commitment, the lower is the tax, the more worthwhile is the
wage, and the more likely one is to take the job. The overall
implication is that technological change rewards individuals with weak
ethical convictions relative to individuals with strong ethical
convictions, and therefore exerts a direct, material pressure on the
evolution of ethics within society.

There are three key, testable implications we would expect to observe if
this model indeed captures important aspects of the historical record.
First, during times of ethically transgressive technological innovation,
the first workers will be individuals who before the technological
innovation were already less ethically scrupulous on the relevant
ethical dimension. Second, workers of the new technology will be paid a
premium which would be, in principle, distinct from any skill premium,
although of course these can be correlated. Third, if we add the
assumption of modern capitalist institutions, wherein wage workers have
no common land and no immediate means of production, then the price of
integrity for committedly ethical individuals literally becomes death.
This last implication means that, if technological change is ethically
biased as I hypothesize, the steady-state to which it must tend in the
long-run is the absolute evacuation of integrity from all participants
in the labor market. Note that this can come about through two different
causal pathways: ethical individuals can learn to sacrifice their
integrity to compete with the unethical and survive, \emph{and/or}
ethical individuals can literally die off while unethical individuals
thrive and replace them (evolutionary change). It should be noted
explicitly that while observations 1 and 2 show how ethically-biased
technological change reward certain ethical orientations differently and
unequally,

Anecdotally, a general acquaintenace with the contours of modern history
seems to suggest some striking evidence for this model of
ethically-biased technology change. Consider, for instance, that while a
certain amount of philosophical liberalism was no doubt necessary for
the industrial revolution to emerge, and this perhaps explains why it
began in England, it seems equally undeniable that the industrial
revolution in turn \emph{generated} an extraordinary instrumental turn
and valorization of self-interest wherever it went. It will obviously
require more work to test this expectation more rigorously, but it seems
highly probable.

In summary, building from insights of the Frankfurt School, I provide a
specific, empirically testable mechanism which accounts for the
large-scale historical variations in ethical outlook they described to
the dismissal of the mainstream social scientists of the time. Working
heuristically from rational-choice economic theory, my stylized
theoretical model is consistent with but more general than the
rational-choice framework because the model here in some sense explains
what rational-choice models merely assume (the prevalance of
instrumentally rational dispositions).

If correct, the long-run, world-historical implication of this model
could not be more staggering for understanding the ethical experience of
living through modern, let alone contemporary, capitalism: if this model
is correct, then it would mean that today we have inherited a world in
which ethical integrity has been subjected to a roughly 300-500 year
forced evacuation by labor market institutions.

\subsection{If educated progressives are so smart, why is right-wing
populism
winning?}\label{if-educated-progressives-are-so-smart-why-is-right-wing-populism-winning}

Moderate educated progressives across the West are currently horrified
by the recent successes of right-wing populist politics, from Donald
Trump in the US to Brexit in the UK. But underneath this public
performance of shocked horror coming from the typical cosmopolitan
progressive today, hides a curiously under-discussed paradox. If the
forces of ignorant hate suddenly appear to be winning, then shouldn't
the sophisticated and morally enlightened be clever enough to understand
what is happening and do something to stop these trends? If we are so
smart and moral and our understanding of the world so much more
reasonable and broad and sophisticated, then why are we so puzzled and
terrified at the election of Donald Trump or the vote for Brexit?
Shouldn't we be able to understand it and act with our greater moral and
empirical sophistication to resist and reverse these developments? If
close-minded and hateful ideas and actors are gaining political ground,
then why are we not clever enough to understand what is happening and
act in order to change it?

I believe the most likely answer is that educated progressives are
significantly less clever than we pretend to be; indeed, if one is
really shocked and horrified at the election of Trump or Brexit, the
truly clever and ethical thing to do would be to think really hard and
honestly about whether one has possibly been deeply mistaken about
something, somewhere along the line. Of course, almost nobody does this,
largely because those whose job it is to think and speak in public get
paid (in money or else cultural capital) to assume the position of
knowing more than someone else. There is no market for people who want
to reflect at length about the ways in which they have been dumb, so all
we ever hear or read is educated progressives pretending to be smart, no
matter how badly they seem to be losing their battles and misestimating
events. Shock and horror are natural enough responses, but that's the
problem: shock and horror are the easiest and most immediately
reasonable responses to any turn of events that reveals something about
one's basic worldview to be mistaken when that basic worldview
nonetheless remins tightly held.

I think the real reason educated progressives are so confused and
terrified is that the politics of educated progressivism is itself one
of the key drivers of right-wing populism and, potentially, fascism. I
will try to explain why but, at the outset, it is crucial to see how
this very possibility is almost by definition unthinkable in the
worldview of the educated progressive, whose most charasteristic mental
and political operation is to use their educational resources to
navigate themselves into those positions that are the most correct, the
most true, the most good. Whether they are conforming themselves to the
correct and the good, or conforming the correct and good to themselves,
is typically undecidable to an external observer; whichever the case may
be, the crucial point is that the educated individual is able to do this
more effectively than the uneducated individual, and it helps them get
ahead of the uneducated, both economically and socio-psychologically;
but if, in the final analysis, the educated have made certain systematic
mistakes, it becomes an especially dangerous problem for the long-run
dynamics of a political system. Such a problem is especially dangerous
because the educated are appointed by institutions to be the chief
intellectual error-correctors; if the intellectual error-correctors are
systematically erring, then the only mechanism for the political system
to correct itself will be through the brute force of those individuals
and groups who have no resources or opportunities to sublimate their
political interests into respectable language games.

It is the educated who, without knowing it, blinded by their overly
effective skills in manipulating symbols, have increasingly made the
world in their own image but otherwise uninhabitable for those without
access to the knowledge economy and cosmopolitan sociability.

The political style of educated progressives is: convert your personal
interests into a moral language that you believe is good for everyone,
then do everything you can to make that moral language dominant, even
against anyone who disagrees or does not understand you. A key point is
that this style \emph{is brute force}, it is simply brute force
sublimated to the ``higher'' level of language, symbols, and morals. It
is brute force because, often quite explicitly, educated progressives do
not care if some people disagree or do not understand; typically,
educated progressives proudly proclaim that the opinions or ignorance of
bigots (defined as those who do not know, understand, or agree with the
preferred vocabular of educated progressives) \emph{should} be
overridden by collective political power. The magic of this style is
that educated progressives \emph{sincerely believe} they are not
operating with brute force but are simply effectuating the triumph of
reason and morality.

The educated progressive gambit works for a while, and educated
progressives get to feel like they are living at the frontiers of moral
historical progress, but the very rope that educated progressives use to
pull themselves above ignorant bigots is the very same rope with which
they suddenly find themselves being hung. And that's because you cannot
effectuate real historical progress of any kind whatsoever by using
unequally distributed resources (education) to promote your own
interests (international jobs, working in the culture industries, etc.)
while proudly dismissing the thoughts and feelings of those you find
repulsive in large part because they do not have the resources to
dissemble their own self-interest into respectable language as you do.
If this is your method of advancing social justice, you're actively
begging right-wing demagogues to rise up, because your entire method
involves using your educational privileges to push those you call bigots
even further and further away from being able to promote their interests
through non-violent channels. This in no way justifies or apologizes for
fascists, it only seeks to give an honest and non-self-serving account
of the real causes behind such awful political formations.

Many educated progressives, openly or secretly, really do not want
uneducated right-wing people to have a public voice equal to their own.
Their intellectual style demonstrates this quite undeniably. But
educated progressives have made the mistake to believe this can work
indefinitely. Perhaps educated progressives have begun to really believe
that uneducated conservatives with so many unsavory notions are
sub-human. Well, it turns out no group of human beings will tolerate for
very long having their thoughts and feelings and interests \emph{defined
in advance} as outside the bounds of acceptable discourse. Any group of
human beings treated this way will eventually prefer to adopt an ethos
of violence against any and all untrusted out-groups, rather than try to
play a peaceful language game the rules of which are quite literally
determined (and frequently, aribitrarily changed) by those with opposite
interests.

We have underestimated the violence of the educated progressive
political style; it has been at once an ethical and strategic mistake,
and now educated progressives are losing because of it. For educated
cosmoplitan types, it feels natural that we are all part of one world.
But people do not automatically have access to the world, or even a
world. Worlds are constructed. It is the implicit claim of liberal
cosmopolitanism that we are in one world. Ergo, if for any reason one
cannot participate in the norms of liberal cosmopolitanism, such a sorry
soul is for all intents and purposes not in the world. To the degree the
liberal cosmopolitan view is shared across a society, someone who for
any reason is not participating in that world can and should feel as if
they are literally excluded from the world. So before we find ourselves
shocked and horrified at some future triumph of violent right-wing
politics, let's also think harder than ever about the possibility that
perhaps we are the ones who have become so skillful at the subtleties of
political violence that we honestly can't believe it.

\subsection{Social science and the radical politics of not
knowing}\label{social-science-and-the-radical-politics-of-not-knowing}

The amount of bullshit circulating at the moment is astounding. And to
be clear, it appears to be just as bad in left-wing circles. In fact,
what I see happening on the left is the most troubling to me because
that's where I'm positioned. There's this idea you should ``punch up''
or focus one's criticisms on one's ``enemies'' but I think that's a
fatally mistaken notion. If you and your friends are thinking or doing
something incorrectly, that is the most urgent issue.

As a political scientist, the truth is I usually don't have that much to
offer regarding current affairs. I think most social scientists, if they
are being honest, have to admit this with respect to most issues at most
times. But if there's one thing my ``expertise'' gives me, if I have one
valuable thing to offer in a time of crisis, it's a highly refined
bullshit detector. If there is one thing you learn as a well-trained
social scientist, it is this: it is so hard to make correct inferences
about what is going on in social phenomena. Most of the training of a
social scientist is learning all the reasons why you cannot make certain
inferences. So in times of crisis, when most people seem over-eager to
make inferences (as a way of dealing with all of the cognitive and
emotional anxieties), it is perhaps here that social scientists are most
useful, to remind you that, whatever you think is going on---you are
mostly wrong.To be clear, when I say most people are ``wrong'' about
most of their inferences, I don't mean that nobody ever gets anything
right, or that nobody understands anything. We all know a great deal,
but it's mostly embodied, practical knowledge. We know not to put our
hand in a fire, and a million other important things. But when our mind
starts trying to identify causal patterns in a hyper-complex situation
(and really all social phenomenon are hyper-complex), collectively we
will generate thousands of hypotheses and most of them will be false.
Some will be true, but remember that some would be true even by
accident. Monkeys typing on a keyboard long enough would produce true
statements in some portion of the text.

Recognizing our incapacity to know things shouldn't be distressing or
disempowering; it's humbling, liberating, relaxing, and empowering. It
reminds you that the little ball of fat in your skull is actually a
pretty faulty device and it's not really your job to figure out
everything going on in the world. Nobody can do that, but a lot of
people think they can (and should); if you think you have this
responsibility, not only will it drive you crazy but, as I said, on net
you will not actually be contributing or helping anything. Again, don't
get me wrong, I think everyone has a lot to contribute---but not in the
form of objective explanations of what is happening in the world. We
have this ridiculous, faux-democratic notion that everyone is entitled
to their own reading of what is happening, but this is wrong. We are all
equal, but if anything, I would say we are all equally disentitled to
our own readings of what is happening---we are disentitled by objective
reality, which is ultimately chaos, and which does not allow any of us
the privilege of knowing exactly what is happening or what is causing
what. I think we can find a radically more true, honest, and ultimately
connective/solidaristic community in the shared realization that I don't
know, you don't know, but we both know we have each other in this
moment. Crucially, you can adopt this attitude in good conscience as
well, because it's nobody's moral or political burden---not even social
scientists'---to save the world or a country or a people by pretending
to have knowledge nobody can haveWe are seeing right now the
extraordinary mass-delusional implications of a media environment in
which every agent believes they are capable of understanding what is
happening, there are cultural and often monetary incentives for
pretending to know what is happening, and no mechanism for sorting true
from false.

The primary problem isn't fake news or purposeful deceit; the problem is
massive new injection of noise in the system, everyday cognitive biases,
and perverse incentives to perform knowledge where there do not exist
mechanisms for testing and sorting knowledge claims (and I would add,
absurd Western notions about personal control and responsibility which
were temporarily useful in early modernity but are now leading to a kind
of mental heat death in the context of the information age). One of the
other reasons an academic social scientist comes in handy here is that
we do not primarily get paid to make prognostications about what is
going on in the present moment. Sometimes people think this makes us
``useless,'' but indeed our ``uselessness'' is what makes us useful in
times of uncertainty, deception, and mistrust: it is precisely because
we generally don't care about pretending to be useful that if we feel
compelled to comment on current affairs, if only to say it is impossible
to know something with any confidence, it should be relatively more
trustworthy than someone who gets paid to provide useful commentary on a
daily basis. In other words, the uncertain offerings of an academic
social scientist are more likely to be a signal and articles by
professional commentators are more likely to be noise. There is
certainly a new cottage industry for academics who wish to enter the
culture market of disingenuously over-confident inferences, but our real
value is that generally if we are shooting from the hip with little to
gain or lose, then you should be able to trust the academic social
scientist, relatively.

I would ask you to remember, especially if you are passionate about
contributing to politics, that false answers are typically more
responsible for evil than honest admissions of uncertainty.We have to
remember that the human mind has evolved to find patterns, even where
they don't exist. This is because, for the greater part of our history,
if there was a snake in the grass and we failed to identify it, we could
be fucked. But if there was not a snake in the grass and we thought we
identified one, no big deal. So we evolved to err on the side of
identifying patterns even where there is nothing. But what's useful for
avoiding snakes may very well be collectively suicidal for avoiding an
infinite set of possible global threats via the internet. Right-wing
people do this with crime and terrorism but left-wing people are doing
this just as badly with the new semi-global, right-wing shift. As we now
have screens that fling unprecedented volumes of noise at us all day and
night (and which allow us to fling noise back into it!), I think we are
really underestimating the degree to which our highly faulty human
cognition, combined with our individual incentives to perform knowledge,
can generate extraordinary harm to individuals and groups, sending
collective understandings down systematically erroneous and divergent
paths, and ultimately shaping actual behaviors of masses of people. And
when the behavior of people is based on any degree of systematic error
that is not being corrected over time, this is arguably the most potent
recipe for almost all of the worst historical disasters.

To put it yet another way, even highly educated and otherwise
trustworthy people right now are doing what social scientists call
``overfitting their models.'' In other words, developing theories that
can fit all of the data they are observing, without realizing that a
great deal of that data is noise. The thing is, a good explanation of
noise is a really bad explanation of reality; what this means is that if
you act or behave as if such explanations are true, almost by definition
it will produce consequences other than the ones you are hoping to
produce.Again, this should all be liberating and relaxing to reflect on.
If there is honestly a lot of uncertainty, and one honestly does not
know, then one honestly deserves to try and relax, pay attention, learn,
think, consider possible hypotheses, update them as you go, and in the
meantime patiently focus on what you do know (inner convictions, empathy
and solidarity for the people you encounter, etc). You are not obligated
to go ``do something'' or ``say something'' immediately if the actual
reality is such that really you are just scared because you don't know
what is going on.Of course, be vigilant, be courageous, say and do what
you believe in, but radicalism is an all-or-nothing proposition. If you
want to be politically radical, you better also be radically honest,
radically humble, and radically transparent. All I'm calling for is
intellectual honesty regarding uncertainty. I'm not saying anyone should
dampen their convictions or compromise with anything they find unjust.
I'm just saying there's nothing radical or even defensible about
effectively making shit up because you want to produce some consequence,
whether it be the soothing of your own anxiety, the production of
``hope'' for others, or the recruitment of others into your group. One
of the most radical things you can do at any time is be correct. And in
highly uncertain times, the most correct diagnosis of many things will
be ``we do not know.'' You can still maintain deeply held convictions,
and act passionately on various projects, while also maintaining the
basic self-discipline of trying to honestly separate signal from noise.
Speak and act decisively, at the highest intensity you can sustain, but
only on the most correct possible interpretation of information. This is
where I think social science converges with the most radical,
progressive politics.

\subsection{Forget it: The dubious moral politics of
awareness}\label{forget-it-the-dubious-moral-politics-of-awareness}

Since the Holocaust, we have been told to ``Never Forget'' more
atrocities than anyone can remember. Given that the
technologically-aided memory capacity of global society has expanded so
dramatically in only the past two decades---hardly a nanosecond in world
history---it may be time to question this moral wisdom increasingly
belied by its ubiquity. At the root of this ethical commonplace is the
assumption that the injunction ``Never Forget'' is logically followed by
its declarative counterpart, ``Never Again.'' In other words, the
positive relationship between memory and awareness on the one hand, and
social progress on the other, is merely assumed. Interestingly, a
formidable philosophical tradition suggests the opposite may be true,
but ultimately it is an empirical question. Luckily for us, as the world
now has small armies of knowledge workers maintaining for their
audiences perpetual collective awareness of atrocities past, present,
and even future, it offers us a kind of natural experiment for observing
whether the will to remember is as closely allied with the progress of
social justice as many assume. Is it possible there might exist a
threshold of social awareness beyond which a remembering individual's
net contribution to social progress turns negative? And if it is
possible that the globally super-charged and market-driven circulation
of injustice awareness is pacifying our capacities in still poorly
understood ways, then is it possible that today one of the most
radically progressive gambits may very well begin with forgetting it
all?

The educated cultures of the global north today are possessed by a
certain \emph{will to remember}---as if it is an obvious ethical
pre-requisite of fighting for justice to be obsessed with information
about injustices. But whence comes this peculiar will, and what exactly
is its philosophy? The most charitable case one can make for it is that
it is about the importance of awareness in the maintenance of justice.
We must always be cultivating the most true, accurate, and balanced
picture in our individual and collective minds, a rigorous accounting of
all of the good and the bad, but especially the bad, so that fair
historical redress may be served and that we may know evil well enough
to prevent it from recurring in the future.

This perspective is reasonable enough, but when one watches a large
number of individuals share horror after horror, over the course of
several years, often about the same topics --- is it not fascinating
that some people come to specialize in curating particular genres of
non-fiction political horror? --- it becomes hard to believe that such
individuals are operating under the belief that every extra ``share''
will decrease the probability of stopping the particular horror in which
they specialize. It seems more likely that such individuals are simply
addicted to the consumption and transmission of information about a
particular genre of horror. Of course that is their prerogative, and it
is as legitimate and reasonable a technique for coping with our human
predicament as any, so I have no interest in shaming or stigmatizing
this particular style of living among horrors and evils. I only wish to
query when, where, and how anyone ever began to think that frenzied
exposure to injustice information might be a method for stopping
injustice from occurring in one's society.

The naïve empirical model in which raising awareness is supposed to lead
naturally to a reduction of the Bad Thing in society, was never to my
knowledge seriously developed as a thoughtful and serious model of how
attitudes and behaviors, let alone social structures, change --- as if
people first became convinced of this model and then chose to delve
obsessively into consuming and sharing horrors. It seems more likely
that a segment of the population who, because of their own history and
personality, prefer to cope with internet-age info-glut through publicly
shaming wrong-doers, happened to connect with an equally large segment
of the population whose preferred way of coping is of the masochistic
variety. In this positive symbiosis, the culture of hyper-awareness as
activism emerges only as a self-flattering and wishfully-thinking motive
imputed to what is simply an evolutionarily selected way of
surviving---that is, continuing---the particular equilibrium of the
social and political status quo.

One good piece of evidence in favor of this interpretation can be found
in what economists call ``diminishing marginal returns.'' One might
believe that informing someone of a horror they did not already know
about is indeed a useful and mutually empowering, progressive speech
act. Maybe even sharing that piece of information a second time might be
a useful way to make sure the message was definitely heard by everyone
in one's audience. But it is in the nature of information itself that
every time a piece of information is repeated, it contains less and
less, well, information. Famously defined by Gregory Bateson as ``a
difference that makes a difference,'' information is information only to
the degree that it sparks in a receiver something not already sparked.
Beyond a certain point, by definition the circulation of information
about horrors cannot be justified by the necessity of spreading
awareness simply because every additional share brings diminishing
marginal returns to the production of awareness. The degree to which
certain people can share certain classes of information so frequently
for so long, testifies to how little this phenomenon has to do with
information and how much it must have something to do with the
compulsion to share.

On closer inspection, then, what appears to be a model of ethical
behavior based on a certain assumption about how the world and social
progress function, may actually be a contingent configuration of two
very particular ways of being ill, which, because of the satisfactions
it gives rise to and the scale at which these satisfactions can be
generated, is simply given a presumed empirical model naïvely back fit
to legitimate and secure these interlocking symptoms. The will to cope
and survive is certainly fair enough, the problem is that it begins to
teach others, and the next generations, that this is not just a way of
coping with the existence of evil and horror within the human condition,
but that it is actually a model for ethical and political goodness. This
is the arrogant and debilitating collective self-harm that I would like
to target in the politics of political information sharing today. This
is why, in a certain sense and under certain conditions, it may not only
be defensible but ethically courageous and politically progressive to
re-learn with a new vengeance the old fine art of forgetting.

It is likely that the will to remember became a feature of modernity in
that swirl of eighteenth-century rationalist energies we now call the
Enlightenment. But it was probably Rousseau who first laid the basis for
a modern opposition to memory as a socially enforced imperative. In
eighteenth-century France, the Encyclopédistes were setting out to
collect and organize all of the world's knowledge in their proud belief
that the promotion of understanding and rationality was an obvious path
to social and political progress. This was also a time when the key
sites of cultural and intellectual life were still the famous salons.
The salon was all about measured and quick-witted sophistication;
participants would memorize cleverly crafted repartee in anticipation of
the buzzwords with which they would be confronted. It is hard not to
note the equivalents in contemporary culture, where as enlightened
netizens we pride ourselves on having all the world's knowledge at our
fingertips.

Similarly today, one can usually predict quite well, simply by Facebook
circulation, which world catastrophe one's dinner party will expect one
to know and care about on a particular evening, allowing us to recite
lines we have already practiced online. Indeed, it is increasingly clear
how much web content comes from a can of memorized repartee assorted and
measured to fit the various genres of catastrophe, as if people wake up
each day hoping to find the persistence of the horror in which they
specialize, given how carefully they have prepared for it to never end.
This is especially true and troubling with respect to the rise of paid,
career writers whose ``beat'' is one of our diffuse and deeply ingrained
types of socio-political evil, such as patriarchy or white supremacy. We
now have relatively large armies of commentators who are in some crucial
sense, no matter their most sincere convictions, invested in the
persistence of certain deep and difficult social problems. Of course it
is not that this army of content-producers caused the problems of their
focus, my point is only that the way social problems arrange us
psychologically and behaviorally, especially in a market society, may be
capable of blocking the dynamics of personal, interpersonal, and
ultimately social change for which one may sincerely be fighting.

The incapacity to forget oneself as a socially respectable
problem-solver leads one to repress in oneself and others all of the
psychological and behavioral needs or desires of the human animal which
do not look like the socially-sanctioned appearance of problem-solving.
But because deep social problems are almost always systemic---that is,
they are usually an inextricable feature of larger patterns which define
our most basic sense of what we call reality---whoever commits too
completely to performing moral goodness comes to function as a law
enforcer of the intellectual and emotional dimensions of the status quo.
In his own language, Rousseau made exactly this argument, and it was
demonstrated most convincingly in the waves of historical change which
would follow his life.

Because he was a provincial from Geneva, Rousseau knew he would never be
able to match Diderot or Voltaire in their vain performances of
memorized alienation. He admitted he had only ``a little memory''
(``\emph{mon peu de memoire,}'' he called it). He tried reading
selections of the \emph{Confessions} to Parisian salons but they fell
flat, probably because one of his main points was to explore
psychological realities repressed and denied by the political ethics of
superficial \emph{politesse} and disingenuous \emph{honnête}, precisely
the ethics of the salon. Instead, Rousseau's intellectual strategy was
to embrace his poor memory and run with it, turning over the tables on
his way out, as it were. ``A state of reflection is a state contrary to
nature, and\ldots{} a thinking man is a depraved animal,'' he would
famously write in the \emph{Discourse on Inequality}. Rousseau was
mercilessly castigated and ostracized for such scandalous pronouncements
against the march of cultivated rationality; he suffered a kind of
social punishment most contemporary intellectuals in the wealthy liberal
countries rarely, if ever, risk. But his disregard for what the
well-bred were busy remembering was at the core of everything that made
his philosophy and life concretely and empirically revolutionary and
progressive, as he became an explicit inspiration in the French
Revolution and the entire intellectual and ethical tradition we now call
Romanticism.

If it was Rousseau who taught modernity the perniciousness of
cultivation and the political power of disregard, it was Nietzsche who
sketched the most vivid portraits of what might be the polar opposite of
the pious memorializer. In contrast to serious and all too human
students of history, Nietzsche taught, individuals who act toward
important goals with passion and commitment are exceptional primarily in
their capacity to somewhat sociopathically ignore whatever would
threaten to interrupt their motion forward. In the \emph{Untimely
Meditations}, Nietzsche writes, citing a phrase from Goethe, that the
person who acts is, ``always without a conscience, so is he also always
without knowledge; he forgets most things so as to do one thing, he is
unjust towards what lies behind him, and he recognizes the rights only
of that which is now to come into being and no other rights whatever.''
In this view, living so as to be a good person in good conscience
becomes an ethical horror in itself, for it is effectively a willfull
commitment to changing nothing plus the dishonesty of retroactively
labeling this paralysis a passion for social justice.

But note how this is not to invoke such cynical mantras as ``you have to
get your hands dirty in order to get things done.'' Most people who
speak such phrases are in bad faith, disingenuously exploiting the fact
that action implies injustice to beautify whatever ethical atrocities
they tend to commit. When a politician spends an entire career accepting
bargains from devils in order to maintain their own power and deliver
crumbs to their constituency, and then they package their moral
degeneracy as heroic sacrifice of their privileged idealism, this is not
Nietzschean realism. This is a much older lifestyle practice for which
we have a different technical term. It is called lying. They are obeying
status quo values, entering status quo positions, to receive status quo
rewards (denominated in status quo values) \emph{through} their offer of
crumbs to constituents. When they claim to be making the difficult
choices required of them, they are merely exploiting a simple fact
nobody can ultimately deny and on the basis of which we already forgive
our politicians in advance---the idea that effective action requires
some injustice---to define their self-interest as a public good.

Memory creates a circuit between possibility and perception; it reduces
the infinite possibilities of what a perception might turn out to be, to
that which it is most likely to be, based on past experiences.
Obviously, this serves a crucially valuable function for the human
being, for it reduces the unfathomable complexity of an infinitely rich
reality to a simpler, manageable number of possibilities which we are
able to learn and conquer. This conquering which memory permits has
allowed for the survival of our species, but in an era where now
thinking subjects are the ones who are being conquered by economic and
political institutions, memory ceases to serve the survival function
that was its \emph{raison d'être} and becomes little more than a
sado-masochistic ritual to help pacified populations pass the time. When
we consider the basic psychological function of memory but apply it to
events over which atomized individuals have no control, hyper-awareness
of atrocities can do nothing other than conserve and ultimately enforce
them as the most likely of an infinite number of possible, virtual
futures. Memory has come to function as a prohibition on the collective
creation of futures in which these atrocities will not exist.

It is not a matter of valorizing fantasy over hard-nosed empirical
realities, as if we have the right to a greater quotient of delusion
simply to make life more tolerable; it is rather that we achieve the
most comprehensive faithfulness to empirical reality only through a
certain distance from mere facts. The obsession with facts and events
over-attunes us to certain empirical data and under-attunes us to other
empirical data, namely the equally real facts of our immediately
physical, emotional, and psychological lifeworlds. While the overall
balance of effects is still debated, it is increasingly well documented
that contemporary internet culture has certain deleterious effects on
our personal and our social psychologies and behaviors. The point is not
to add yet more sweeping assertions about the effects of digital culture
on society, it is rather to highlight how, for the netizen passionate
about social justice, the actual and currently on-going effect that
contemporary hyper-awareness is having on millions of people is, for
some reason, not admitted into the orbit of their seemingly passionate
concern.

The reason, I hypothesize, is that social justice netizens are exactly
those who have most fully forgotten how to forget, for how else could
this rather peculiar way of relating to the world become for some an
almost full time vocation? An awareness so saturated with facts of
horror that it is unable to take flight from the weight of history even
enough to feel and observe what is happening to oneself and those in
one's immediate vicinity. Or worse, a heavy awareness so crushing that
even when one periodically feels and observes what is happening to one's
own relation to the world, one no longer feels oneself worthy or
deserving of becoming more vital and joyous one day in the future. If
this sounds harsh, I am able to write it only because I can identify it
in myself as much as I observe it others.

I have a friend who spent a recent weekend volunteering at the refugee
camp in Calais. They have gone a few times now in the past few weeks,
making the daylong drive from the south of England and staying at the
camp for several days at a time. I happen to be in a text-messaging
group with this friend and a few others. On that Saturday, while my
friend was busy trying to save lives, I sent this group a really funny
joke. No, really, it was extremely funny. Of course my friend trying to
save lives was unable to respond to my frivolous delight. It immediately
occurred to me that such a meaningless amusement was criminally childish
of me given that I too could have been out trying to save lives with my
friend. My weekend was free and I was consciously choosing to idly enjoy
it instead of helping to save lives. In some terrible sense, I was
choosing to let the refugees suffer. Of course there are plenty of less
offensive ways to rephrase this fact, but I am not sure they would be
any more honest or politically useful than all the retrospective
rationalisations I have already rejected above. If one possesses a
paralyzed over-saturation of awareness, as I suspect many of us do, then
one would need to forget a certain amount of suffering in order to live
a life that might one day become worthy of that suffering. One must pass
through to the other side of suffering rather than organize oneself
around it simply because one cannot sit still in its proximity. To live
up to others' suffering, one must live freely and fearlessly in the
first place so as to have something meaningful to expend toward a
radically different world to come, but now. I do not believe refugees in
Calais need the presence of my overworked, alienated, exhausted friend
to help their prison function somewhat more humanely so much as they
need us, and other residents of the wealthy countries, to sincerely
become human enough that nobody from our ranks would be capable of the
highly refined sociopathy required to repel refugees from a national
border at gunpoint.

Again one notes a certain perversity in the distribution of memory and
sociopathy. My dear and blessed friend who cannot \emph{not} try to help
the refugees in Calais is incapable of the mild sociopathy of enjoying a
relaxing weekend because they are highly faithful to their memories of
human suffering; this absorbs their resources for flourishing in their
own life and they call it working for justice; with institutional
support the border police confer to themselves an extreme sociopathy and
an absolute forgetting of suffering, which also destroys their humanity,
and they also call it working for justice. What if the real problem here
is something so inane that we cannot remember it precisely because the
public sphere is so heavy with bad memories that the light ones fear
punishment for playing? I have begun to believe that the real, core
problem in this example is not the refugee crisis but the fact that my
friend and the border police are so willing to work on the damn
weekends. My friend should be with me so I do not have to text them poor
substitutes for intimacy, so they do not have to feel so anxious about
whether their lonely heart is doing enough good for the world, and so
that we might one day become so collectively alive and healthy and
bonded that we might just start tearing down fences simply because we
forgot why they are supposed to be there. And if all the border police
cared enough for themselves to take the weekend off, they might just
gain some real memories of their own --- as if they would pick up the
ones dropped by my friend if only my friend could let go of a few. They
might remember, for example, that perhaps they always had been planning
to quit that job, anyway.

\subsection{When the world seems to
worsen}\label{when-the-world-seems-to-worsen}

I cannot pretend I know how to live the most true and just life, but as
many dimensions of contemporary politics appear to worsen ---~from
fascist and nationalist movements around the West to the increasing
visibility of the racist police state of the United States ---~I am
dumbfounded by how many times a day I encounter people who are very
clearly only pretending to understand what is happening in the world and
what others should do. Whenever the affairs of the world seem to take an
unexpected turn for the worse, one seems to find an uptick of this
widespread, knee-jerk repetition compulsion: to deal with political
horrors and uncertainties by increasing one's outward appearance of
confidence and righteousness, while silently grasping to secure one's
emotional and material interests however one can.~The more I observe
this, in personal encounters as well as in mainstream and social media,
the more likely it seems to me that this tendency is itself one crucial
reason why everything seems to be drifting in such a horrible direction
while nobody seems to have any especially convincing advice for how one
should live so as to maximally help redirect it.

Given this view, the last thing I would try to offer you is yet another
disingenuously confident and righteous diagnosis of the world's
happenings. Yet there remains a tremendous amount of room for all of us
to think about and articulate within our own lived relationships that
which we know to be false about the world today (a task very different
than pretending to know what is true or exactly what anyone else should
do); to try and fracture and chip away at all of the diverse forces that
keep most of us, myself included, perpetually locked into daily routines
where we know deep down inside that we are not living as fully and
correctly ---~that is to say, as justly and dangerously and rebelliously
---~as we know we could if only we knew how and had the necessary
emotional, physical, and material resources. So what I would rather try
to offer here is a reflection on this easily forgotten reality; a
reminder that nobody is ever going to save us, but that if we pull out
all the stops to do so, we can and might very well save each other. We
can teach each other and give each other everything we need, but almost
everything in our society exists to make us forget this and to block it
from developing when we remember it. One can (and I think, should)
decline the path of arrogant intellectual authority while still having
the confidence and courage and zeal for the truth that leads one to
pinpoint and share where and how all the dominant blockages and false
paths occur around oneself; how to spot them, how to avoid them, and
together how to create new but currently impossible tactics for cracking
them and co-creating actually real, shared lives in the process of
liberation. For collective liberation is not a far-off, possible stage
of history, it refers to the concrete mental and physiological movements
that sweep through any collection of bodies when those bodies are
learning what is false and updating the entire course of their life
accordingly. Of course there is plenty of valuable information and
analysis available out there, but very few intellectuals or writers
today are willing to seriously acknowledge the fact that nourishment for
real processes of collective learning and liberation are almost strictly
prohibited from public communications today; there is simply no oxygen
for them, as infinite messages compete so ruthlessly for the shortest
attention spans ever to evolve in the history of the world, and mostly
for the explicit goal of monetary or cultural capital. But in this
massive blindspot of popular conversation lies a vast reserve of real
gold for all people today interested in thinking, writing, and living
intensely for and toward radical social justice for all as soon as
possible.\\
\hspace*{0.333em} It is with this in mind that I write you today;
perhaps my reflections will help to puncture and deflate any nonsense
judgment or opinion you have come across in the past few weeks which you
felt to be wrong but by which you were still weighed down. Perhaps you
will take it upon yourself to seize some original insight you possess
but have not bothered to write or express because you knew it wouldn't
sell on social media. Perhaps you'll share it with someone you know and
from there new modes of life will emerge. This is not hippy-dippy karma
talk; all unhinging from falsity and exploitation in favor of radically
attentive co-presence, expression and creation with others is concrete
revolutionary movement, plain and simple. Such phenomena have swept away
entire nations and epochs. We need to find that place where we are no
longer affected by the over-confident judgments of idiots whenever our
sincere examination cannot find real learning or nourishment or
empowerment from them. That place where we could not not~say or do what
we believe. I write to find and strengthen the reality of that place,
and I share what I write to make it real for myself and anyone else who
finds truth in it. I want to support you or anyone else trying to do
that too, even if I disagree, for if we are being truly honest than we
would have to arrive at the same place in the end.

In a private letter days after the UK voted to leave the European Union,
I recently wrote to a friend and comrade about the reactions I was
observing in my own circles. Here I would like to share with you
something I wrote there (after some minor edits for clarity):

\begin{quote}
Everyone is definitely losing it lol, I think we all need to chill and
love ourselves and each other and, as ever, figure out how to proceed
from a base of calm joy. Also, reacting is always the exactly incorrect
thing to do, I think. Everything today is reaction, and more than ever
I'm truly and sincerely losing my capacity for this constant
reactiveness\ldots{}
\end{quote}

\begin{quote}
Yea, I think there is something very perverse in our perceptions and
attitudes, something deep and wide, that is in us perhaps even more than
in events--that is making us crazy, or something. It sounds or looks
selfish and self-absorbed to many from the outside, but I have almost
completely stopped reading the news and thinking/caring about all these
external events and I put almost all my attention to how I and the
people around me are feeling. It's not that I think I'm free of
responsibility for what's going on in the world, but I definitely don't
think I'm responsible for all of it! Fuck that. Like I don't think I
have an ethical obligation to know about or care about every bad thing
everywhere. I think that's straight up psychopathology of some kind. So
I think there's this thing where activists feel like they need to
know/care/attend to every horror\ldots{}. then there's the mainstream
moderate progressivism which is not actually rooted in any kind of
deeper radical attitude/being/community so they are just scared and
starting to freak. Which is fine, and natural enough. But that's because
they live a generally deluded and dishonest life and are very happy to
go along most days of their life as if everything is fine--so when major
events touch them in unpredictable and undesired ways, they just get
scared and sad and start posturing and jostling in dissimulated
self-protecting ways that throw up even more blocks to radical revealing
and connection. And lots of radicals are going for that swim too, which
makes me sad\ldots{}\\
\hspace*{0.333em} So yea, I'm OK. To be honest, I don't want to sound
glib, but I want to say I kind of don't give a shit about the EU and the
million things happening around the world. I think it's a pathological
Western thing to flatter oneself with feeling responsibility for it all.
I want to be so based, so in touch with the meaningless of existence and
the absolute meaning and truth of the infinite value and potential of
all human beings, so attuned to what is beautiful, what is true, what is
funny, what is pleasing, what is rewarding, who I love and who loves me,
and our infinite capacities for ecstatic world-creation despite whatever
the psychopathic institutions might continue to throw at us--that bombs
could start dropping on Southampton and I would probably just keep
reading what I'm reading, writing what I'm writing, messaging you all,
making things and making plans, and feeling exactly the same as I do
now, but even more intensely and outwardly, because I like to think that
is what it means to be a revolutionary, to be already radically more
attuned with the truth than normal people (or at least less hooked into
all the stupidities of the day), so that things don't surprise us and
things don't knock us off the sincere search for true living: to already
be horrified and enraged at everything humanity has become and also
perfectly filled with joy and ideas and constantly, nonetheless, finding
or making liberation come closer and closer despite everything, and to
love one's fate simply because it is one's fate\ldots{}\\
\hspace*{0.333em} If one looks at what most ``concerned'' and
``involved'' socially and politically active people look like right now,
if one looks honestly at their lives, how they are speaking and what
they are doing and not doing, one really starts to see that the
orientation I'm describing may really be the only possible model of
revolutionary being today, if by revolutionary being we mean radically
attentive to reality, socially and politically engaged as deeply and
widely as possible, and a revolutionarily prepared/able propensity to
act justly and effectively on all possible environments at all
times\ldots{} the more I try to think this through, the more almost all
currently existing activism really seems like a kind of crazed
simulation of caring and political action that is actually not much more
than the complete overheating and implosion of the Western human being,
I think\ldots{}
\end{quote}

\subsection{\texorpdfstring{Is there an obligation to ``speak out'' if
nobody is
listening?}{Is there an obligation to speak out if nobody is listening?}}\label{is-there-an-obligation-to-speak-out-if-nobody-is-listening}

Whenever horrific, unjust things occur, online social networks fill up
with people telling you it is your duty to speak out against those
things on the internet. But that is wrong: One's duty is to determine as
honestly and rigorously as possible the empirical conditions that would
make those horrible things stop, and then to mold one's life around
increasing the probability of those conditions coming into existence.
Clearly, it is hard to know for sure what exactly are those conditions
and how exactly they are most likely to be brought about, so even in the
most optimistic, mass revolutionary awakening we would hope to see many
people doing many different things in many different ways. For some
people, their honest path may involve social media activity, so that is
what they should be doing. For many other people, however, social media
is the exact opposite of what they would do if they honestly and freely
came to their own personal judgments of how they should live against
systemic injustices.

When social media activists start to think and speak as if social media
is the primary yardstick of radical consciousness and public action, it
is here that online social justice culture becomes a truly disingenuous
and ultimately conservative racket. This is where social justice culture
stops being rebellion and becomes a market of cultural entrepreneurs
selling various diets for their followers to feel morally healthy in an
otherwise morally intolerable world. One of the best signs of this is
when simple empirical observations are so confidently dispensed with;
given how easy it is to see that there are many life-long, deeply
committed revolutionaries and fighters for social justice alive today
who do not appear to be active on social media--to declare that social
media activity is the measure of a person's opposition to injustice is
actually a very clever litmus test to see how many of one's followers
care so much about moral approval from the poster that they are publicly
willing to admit and reveal demonstrably incorrect thinking and
information. Most people who make such obviously false statements are
really saying, ``I am so satisfied with the moral energy or comfort I
have found on social media that it must be morally necessary for others
also.'' This idea is false and conservative, primarily intended to
conserve the existential needs of those speaking it. It is also
conservative for blinding and discouraging others from the thousands of
different revolutionary paths they could and should be off inventing.

Many of the smartest radicals I know are starting to realize that the
truly ground-breaking, revolutionary paths of the twenty-first century
will emerge in collective maneuvers out of this permanent state of
anxious online reactivity. The internet is no longer a medium through
which humans express and coordinate rebellion against external events,
it is the medium through which external events coordinate how humans can
and cannot express their rebellion. Anyone who has a good grasp of
social reality and its mechanisms does not need to know every detail
about every new police murder; when one understands something, one does
not need to know certain details because one is already grappling with
more general levels inclusive of those details. The revolutionaries of
the twenty-first century will move away from weekly details in favor of
becoming doubly based and sane and balanced and even joyous despite
everything, because we realize that today these are precisely the most
rare and most powerful capacities, perhaps the only ones, that have any
chance at what we are up against. In the past year or two, I have
decided to consciously abstain from commenting on current affairs
precisely so I can save my neural and emotional resources for long,
thoughtful, heartfelt reflections to share with people I know and care
about, or to attempt writing for larger audiences than the few dozen
people Mark Zuckerberg allows to see my Facebook posts. I personally
think that's a more revolutionary relationship to modern technology. But
of course I think other people should do what they think is best. In my
view, the point is to become capable of producing but also noticing
revolutionary gestures in more and more possible modes of speaking and
being.

\subsection{\texorpdfstring{``I Think We're Really Onto Something:''
Mark Fisher and My Revolutionary
Friends}{I Think We're Really Onto Something: Mark Fisher and My Revolutionary Friends}}\label{i-think-were-really-onto-something-mark-fisher-and-my-revolutionary-friends}

Mark Fisher was a revolutionary, but I do not mean his writings were
revolutionary (although they were); I mean that Mark Fisher was a
revolutionary in a very specific sense of the word, a sense that does
not necessarily apply to everyone who happens to be sympathetic to
radical ideas or causes. We know this because, over the past two years,
we have not just been friends with Mark. A number of us have been,
together, in the process of concrete, organized, revolutionary political
transformations. These transformations remain somewhat obscure, and
before Mark's death I did not fully comprehend what he and all of us in
\href{http://www.weareplanc.org}{Plan C} have been doing over these past
couple of years. But now, for me at least, Mark's death has been like a
flash of emotional lightning that suddenly illuminates a dark forest
pulsing with life, revealing with undeniable clarity \emph{where one
even is}. In an email, Mark once wrote to us: ``I think we're really
onto something\ldots{}'' I think he was right, in fact I think he was
more right than any of us have known what to do with. With Mark's
passing I believe I can see more clearly now than ever what exactly we
have been onto. To be honest, I did not know Mark well, and I have only
a passing familiarity with his writing; that I have so much to say can
only be attributed to the political processes that have been, over the
past few years, sweeping a few of us away, together.

§

Mark always struck me as the type whose opposition to the status quo was
such that he sincerely thought, felt, and lived as if it could not be
so. For some people, opposition to the status quo can be a form of
adaptation and survival. For these people, activism provides
socio-psychological supports that make the experience of the status quo
tolerable. But others are plagued with a humanity that will not go away
no matter what you offer it, a certain inability to accept the status
quo, an incapacity to integrate oneself into its consistency, resulting
in a kind of maladaption risking rather than securing survival. For
these types, entry into radical politics is not about making life
livable under unlivable conditions, it is about figuring out how to
produce genuinely livable conditions at any cost. This is a subtle but
crucial difference: the former model waters down the definition of what
is considered ``living'' in order to survive and claim we are
``living,'' while the latter admits plainly the unfortunate but real
challenge of an unjust political order: \emph{either} overthrow
unlivable institutions and make life together possible immediately, or
we will already be dead.

In my own view, everything seems to suggest that the truly revolutionary
life today must be of the latter type; it would seem that revolutionary
politics today could not be anything other than a kind of minimally
sustainable, reproducible type of militant maladaption, the capacity to
creatively occupy oneself as something that consciously and purposely
does not belong to nearly all of that which is currently and falsely
called reality. But obviously individual human organisms have limits and
this tendency can lead to self-destruction; one question we have
therefore been grappling with is, how to sustain this kind of creative
maladaption over time, how to make revolutionary maladaption socially
reproducible.

It seems to me that, in his intellectual work, Mark sought actively to
inhabit this heady, scandalous mental space in which everything people
call real is, clearly, not real. Exciting and true, but anyone who has
ever sought to engage in radical intellectual work over a period of time
learns quickly that this parallax is quite a load on the nervous system,
because every interaction in our really existing Boring Dystopia will
require far more emotional and cognitive stress than would be required
if one simply took the Boring Dystopia to be real. Now, if we have any
hope of living a true life together then we must at all cost hold onto
this heady, stressful, critical distance. But I think one thing Mark
understood was that there do exist tactics and techniques for making
true life possible despite everything. I think Mark was a maestro of
such tactics, not just because I got to observe him performing them (as
I will sketch below), but because to do his kind of radical theory over
any period of time. you need them. That he was able to write all of
those words on topics such as mental health and capitalism, in that
dangerous and difficult mental space he was most known for, is evidence
enough that he possessed some mastery of how to power a life that is not
being fueled in the conventional way through complacent, adaptive
negotiation with the status quo.

I should say at the outset that I am not interested in claiming Mark for
any particular thesis or agenda; like any genuine, radical intellectual,
I am sure he thought many different things that he never brought to
perfect coherence. Yet I do believe, for a number of reasons I will try
to articulate, that Mark was especially interested in this question
about the interpersonal and social tactics that transform individual and
group consciousness into weapons that perform concretely revolutionary
work (however slowly and invisibly at first) on even the largest-scale
political and economic institutions.

One reason I know Mark was keen on this point is that he told me so. I
remember one time he was telling me about the most recent book project
he had been working on. I asked him about the thesis. He summarized it
by saying something to the effect of, ''Basically, 1970s socialist
feminism had it all figured out." We can debate what he might have meant
by this, but I believe he had in mind especially the feminist
consciousness-raising groups prevalent at the time. Even more
specifically, I think Mark was interested in how these
groups---dedicated to the sharing and making visible of once silent,
privatized struggles---\emph{really worked}, not just for ``therapy'' or
the now more chic ``self-care'' but as a \emph{bona fide} methodology
for producing large-scale, revolutionary political change at the
systemic level. The various movements of this one particular historical
moment were crushed, yes, but the point is that \emph{it worked,} as far
as it went. Of course, there will be opponents and enemies, but the
basic method is a real, concrete, and reproducible way for even lonely
individuals and small groups to immediately begin the overthrow of
dominant institutions.

My memory of his characteristic, nervous excitement seemed to be saying,
like, ``We already know what to do! Next time, this time, we just have
to figure out how not to get crushed!'' That is, we have to figure out a
number of auxilliary questions that our revolutionary predecessors had
not fully worked out---such as how to expand, aggregate, and materially
reproduce consciousness-raising dynamics against powerful reactionary
forces and agents---but as to the basic nature of revolutionary
movement, its primary source \emph{and} destination as an actual
activity human beings can do, we already know it. It is the concrete,
immanent process of human beings seeking, through each other, their true
consciousness. That might sound woo-woo, but I will argue that the
status quo reproduces itself in large part by making this proposition
\emph{seem} woo-woo. Our fear of being naïve, our fear of wagering too
much on our own immediate shared consciousness---more and more I think
this is \emph{the} enemy, or at least the single most real and vicious
tenterhook that status quo institutions have successfully lodged in our
bodies. It seems to me that radicals and activists today may be
scrambling to find what is already under their noses, in the historical
sense that the 1960s already demonstrated how to produce massive,
global, political shockwaves, but also in the immediate interpersonal
sense that all we need is exactly whoever is right in front of us.

Another minor exhibit. Within the group, I once wrote an essay that
argued
\href{http://www.weareplanc.org/blog/consciousness-raising-is-social-strike/}{consciousness-raising
is effectively strike action}, the real and concrete withdrawal of
cognitive and emotional energy from the status quo. The essay was
critical of many basic assumptions of contemporary leftism and I know
that Mark was sympathetic to the essay. Interestingly, he was very
worried about the backlash I might receive, most likely due to his own
ghastly experiences taking risks on the internet (which I consider in
more detail below). Of course nothing happened, my article received the
much harsher fate of a generally tepid response. Nonetheless, this all
suggests to me that what I was trying to articulate in that essay
overlaps, at least in some degree, with what Mark had been thinking in
recent years. Something difficult and apparently sensitive, something
that progressive folks either don't care about or get very angry about.
It all seems to indicate that we are getting closer and closer to
understanding what exactly we have been onto.

§

When one speaks the words ``consciousness-raising,'' the connotation is
so strongly one of New-Age spiritualism that, from a political
perspective, the conversation is usually over before it starts. I think
the coming years will show this to be an error. Nonetheless, for this
reason, I prefer to speak of the physiological and biochemical effects
of consciousness tactics; how shared consciousness---if all parties
truly take that shared consciousness to be more real than official
reality and allow their future thoughts and behaviors to morph
accordingly---produces concrete attitudinal and behavioral effects that
immanently decrease the power flowing into the institutional center
while increasing the autonomous power circulating in the commune of
those who compose it. Even better, these tactics come with the
exceptional virtue of being immediately palpable in the body and mind
when executed correctly, and so they are self-guiding and
self-reinforcing. Relationships conducted in this fashion become
veritable collective revolution machines capable of spanning vast
distances, but only if they are done correctly. Such relationships can
and will take infinitely different forms, but I think there is perhaps
one hard rule. There will be various implications from this rule,
implications which will have to be identified and dealt with creatively
depending on the situation, but only one hard requirement. In my own
view, I summarize it with the word ``honesty,'' similar to
``conscience'' but more secular and relational, like ``truth'' but less
formal and more pluralistic.

In a nutshell, I would venture a possible definition of
consciousness-raising as interpersonal communications, on any scale,
motivated by \emph{nothing but honesty} and unconcerned with
consequences. By doing this, conciousness-raising is a form of direct
action, immediately available between any two people (or more), that
withdraws one's labour from the status quo and immanently produces what
you are welcome to call freedom, energy, joy, or power. At a certain
resolution these can all be thought of as interchangeable. While this
might sound too simple to be serious revolutionary politics, the truth
is it's very difficult and extremely rare. Consider the extraordinary
fact that such an orientation is almost impossible to find in activist
circles; almost the entirety of contemporary activism is organized
around the pursuit of certain consequences, to such a degree that in
activist circles if your thoughts and speech are not perceived as
contributing to some future consequence, or if you are not minimally
able to produce speech that has certain immediate consequences
(e.g.~making people feel ``hope'' or appearing ``useful'') then you
might as well not even be there.

There is massive problem in the activist instinct to organize your
thoughts and actions around producing consequences (a fancier term for
this is ``instrumental rationality,'' and it is basically the
rationality of modernity and capitalism). The problem is that, in your
attachment to those consequences, you are liable to make mistakes and
tell lies without even knowing it. And once errors or lies are
circulating, however tiny, everything you try to do with anyone will be
doomed. First, it leads to the crucial error that you see other human
beings as means to some end, whereas in fact the truth is they are ends
unto themselves. Humans are not valuable for some purpose, they are the
creators of these odd things called values, and if you think about it,
that is one of the main reasons why we believe all humans must be free
and equal in the first place. But this error is not merely an ethical
mistake that does violence to others, it is a practical political
mistake also because it blocks revolutionary dynamics before they even
have a chance to begin. The whole problem of alienation under capitalism
is that we have all been reduced to objects in a system we have no say
in. We have to learn how to become revolutionary, from the starting
point of having been born as objects, but when we assume that activism
means making yourself an object or instrument useful for the goal of
producing social change, then we are prohibiting exactly what we really
want and need and the only thing that fuels macro social change anyway.

Therefore, it stands to reason that the only possible first step toward
transforming the currently existing social system is to create minimal
spaces, with at least one other person, in which both parties serve
absolutely no purpose. And the only way to create a zone in which all
parties serve no other purpose is by committing to the only criterion
than can possibly attune diverse atomized individuals: honesty. Honesty
converts the most diverse individuals to the only unification that
preserves all of their differences; everyone can be as radically
different as they please, and yet attuned around the only thing they
truly all share, namely, the objective fact that none are objects to any
of the others but all are their own autonomous ends, that all are
recognized as the creators of themselves, ultimately subordinate to
nothing. If this feels uncmofortably ``individualistic'' for altruistic
types, I need only remind you that this only works as a collective
activity, and the truly autonomous individual immediately recognizes
this individuality as a gift of the community. If this feels too simple
or easy to be a serious revolutionary politics, I need only remind you
that this is harder than you think, so accustomed we are to constantly
calculate consequences. Yet it is only in this unique situation of
purposelessness that one can exit the state of objecthood under
capitalism, in order to experience, if only for a minute, what it feels
like to be free. It is horrifying but I genuinely believe there are many
people today who have never felt what I am talking about, because the
constant mental chatter that is constantly calculating consequences has
hijacked our experience of each other to such an extraordinary degree
that we don't even realize it.

Nobody wants to admit their mind and body are so fully hijacked (in part
because people won't like you, etc., i.e.~the consequences), so we all
continue this horrible state of things in which we actively push away
from ourselves and others the only really desirable thing. The other
reason I believe there exist many people who have never really grasped
or cannot remember this experience is that, feeling or even remembering
such an experience \emph{forces one to be a revolutionary.} If you
really know or remember this feeling, you cannot not find yourself
foaming at the mouth in opposition to the absurdity, stupidity, and
brutality of almost everything currently existing under the label of
``reality.'' That the average person appears to at least publicly speak
and behave as if the offical reality is real---that is data supporting
the inference that the very experience of true autonomous existing is
itself going extinct. Or maybe everyone knows it, but we're all too
afraid to truly \emph{speak and act} accordingly. Either way, the upshot
is the same: revolutionary politics, in the first and perhaps even final
analysis, means nothing other than the immanent production of autonomous
communal social power through the basic principle of radical honesty,
which implies immediate de-objectification followed by all parties
becoming \emph{whatever} they are (i.e.~flourishing). By gaining a
collective mastery over this production, how it works and how it breaks,
we expand the commune indefinitely.

It is worth remembering that the world-historical revolution of
capitalism itself, which overthrew feudalism, operated on precisely
these terms. Whatever we might say about the inhumane consequences of
capitalism, the pioneering individuals whose attitudes and behaviors
would lead to generalized capitalist society were: highly creative,
courageous individuals (in the sense of defying social expectations) who
met in new and uncharted zones (the cities), who acted to manipulate the
nature of reality by leveraging new forms of knowledge and new forms of
technology that the traditional status quo repressed. They started in
small groups, sometimes as individuals and sometimes in small networks
of oath-bound individuals. Fearlessness, creativity, trust, and the
purposeful alteration of social reality in a way that no one's ever done
before, produced a world-historical revolution. There's no reason
capitalism can't be overthrown by the same type of operations, this time
geared toward the the truth of our being rather than dishonest material
interests in commanding nature and each other.

§

Most of what most people do generates nothing but their own misery, and
bad faith converts this misery into only a minimally tolerable survival
(and even this minimum appears increasingly hard to maintain). Almost
everything that passes for education today is essentially false. Most
human relationships, at least in the overdeveloped world, range from
empty to shit, as the number of our weak relationships has increased and
the number of our deep relationships has decreased. And even the best
benefits you can get from the status quo---if you're really lucky,
privileged, and/or do everything by the rules---don't even give you that
much security nowadays. These are some reasons why today, it is in some
of our own lived relationships that we see not merely potential, but
rather the site of currently unfolding revolutionary dynamics we are
only beginning to decipher.

I am writing this on the day that Donald Trump is being inaugurated as
the President of the United States. There is something significant in
the fact that I'm thinking far more about Mark Fisher and my
revolutionary friends than I am about Donald Trump and the government of
the United States.

There's this long-standing assumption that educated, progressive
individuals should pay close attention to national and international
news, but if high-level politics and what is called the news are both
institutionally and ideologically locked down to an unprecedented
degree---as I would argue they are---then I believe that today, educated
and progressive individuals will increasingly learn the courage to
unhinge their attunement from what is effectively at this point mere
noise in the social system. I think one of the discoveries some of us
have been making recently is that, when you do this, in conjunction with
doubling down on your attunement to dearest friends and comrades, so
long as they are also honestly attuned to you, then fundamentally new
energies emerge into this new collective entity-machine-project that
feels quite literally out of this world. I don't mean this in a woo-woo,
spiritual way, I mean fundamental physiological, biochemical effects are
triggered that then ripple out into speech and behaviors in organic ways
tending to the overthrow of institutions.

For people who see short and easy proclamations on social media as the
key gauge of someone's political life, I am happy to give you 30 seconds
to type that I think Donald Trump is very bad and, whatever it is, I'm
against it. But I'm conserving my energy for larger projects; for my
living, intimate accomplices and my revolutionary friends dead and
alive. The big center is a massive, empty zone filled with little more
than the fears of those who incorrectly believe there is still something
there. Mark's death is teaching me that, more and more I want to wager
everything on my friends, and that means moving investments away from
the big empty center of this dead society into the spaces, times, and
experiences that you have the concrete ability to fill with power. None
of this is an attack on other styles, it's just to remind everyone that
silence is not always complicity and indeed it is sometimes the mark of
a groundswell you may just not know how to interpret.

That Mark and many of us have been onto something different, ever so
slightly different but crucially, categorically different, is nicely
measured by the reception of Mark's infamous essay on the Vampire's
Castle. First, I think time has shown that essay to be way more correct
than incorrect. I'm very sorry but anyone I know who has half-honestly
watched the sociology of left internet discourse evolve over the past
few years will agree on this point in private. Many remain afraid to say
it, but with Mark's passing this feels more important than ever to just
put on the record. I remember following the whole fiasco when it
happened, before I even knew Mark, and I thought it was absurd but I
didn't dare to say so. That's shameful and embarrassing. Any
self-respecting adult has to call bullshit wherever they honestly see
bullshit, in public, without apologies. I know way too many people,
myself included (although I'm trying to end this), who won't say in
public really important thoughts and feelings they have about various
habits and tendencies prevalent in what passes for radical or
progressive politics today. I won't argue it here, because anyone who
would be offended by what I'm suggesting almost certainly won't be
convinced and most people whose opinions and judgments I know and
respect know what I'm talking about or at least accept and respect my
comradely right to say what I think without apology. And here's where
this gets real: the truth is I wouldn't even be writing this if I wasn't
at this very moment embedded in real liberatory dynamics with others who
I know have my back because they are themselves flying on the same
winds.

A little story I haven't told many people. I have this draft book
manuscript and Mark once invited me to share it with him. He was one of
the first people I had ever shared it with, and honestly it is a pretty
scrappy and highly idiosyncratic project that I could not have imagined
appealing to anyone. But of course he loved it, or at least pretended to
love it. His encouragement could not have come at a better time, it was
a really long and dark period of nothing but rejections and failures on
all other fronts, intellectual and personal. His interest in the book
was maybe my most positive achievement I had in the entire year of 2014
-- 2015. And again, where many people might only see a minor act of
kindness, I think there is something much more substantial, if we can
learn to see it.

Dispensing encouragement to younger people can be a world-transforming
political action. And if there's one thing that emerges from all of the
beautiful tributes that have been written recently, Mark appears to have
done this on an almost industrial scale. I was somewhat humbled to learn
I am not so special, but impressed to learn that Mark appears to have
been on some kind of mission to push forward everyone he possibly could.
And you know what, lots of notable radicals or intellectuals or
academics are nice people and they try their best to be ``supportive''
of others, but there are levels to this. This is where, if you look
closely enough, you'll see that people like Mark are not just kind or
supportive; he was practicing a revolutionary politics much harder and
far more interesting than just being kind.

If you meet someone you admire and they give you some general positive
feedback or words of encouragement, the actual transforming effect is
going to be conditional on a series of other factors. Typically, you
might find it vaguely uplifting and inspiring for a little while. But
when someone you admire goes to the same political meetings as you, and
sits around before and after just like you, somewhat awkward, somewhat
terrified of recent news, and personally, vulnerably desperate to change
everything that currently exists, with you, well you know what? It
changes everything. The effect is totally different, far more powerful,
far more lasting. And it matters when radicals are also respected in
more status quo hierarches---while of course there is so much to
criticize about those hierarchies. When people such as Mark, who could
be off writing cool books or seeking an academic promotion, are going to
the same meetings as you because they genuinely want to make revolution
now, it produces a unique effect. And I think that's because no matter
how radical we are, we cannot help but be affected differently depending
on where a signal is coming from within the social status hierarchy.
Placement in the social status hierarchy should mean nothing whatsoever
for how we value or treat each other, my point is just that when people
possess status quo cultural capital and they are choosing to invest
themselves in the hard work of organized revolutionary politics, this is
something relatively rare and it produces unique effects that deserve to
be appreciated.

It is these types of interpersonal activities that generate irrevocable
anthropological transformations. Mark's interest in my book was not just
``encouraging''---it effectively supported my entire will for almost a
year at a time when so many rejections were really making me wonder
whether I was maybe just dumb or crazy. But also it altered the course
of my life, to make me more invested in the real, immediate
actualization of revolutionary political change, because our
relationship was one defined by a revolutionary organization and if I
felt indebted to Mark's support what that really meant was I was
indebted to keep figuring out how to make revolution. I'm fully aware
how absurd this might sound to others, but think about it. Since that
time, I've had some modest academic success in my bullshit bourgeois
career, which means my precious ego and income are pretty secure at the
moment, so this is exactly when most people start to drift from their
youthful radical politics toward a comfortable integration with the
status quo. I have every social, financial, and cultural reason to now
just kick back and enjoy my permanent academic post. But now \emph{I
can't do that,} and I'm happy I can't do that, but the reason is because
through my revolutionary friends I am increasingly and irrevocably
indebted to figuring out how to make revolution---to pursue my own
liberation means pursuing the liberation of those others who are the
concrete, direct generators of the power that has animated me over the
past two years.

This is what we are onto. True attention and care, radical honesty and
making shared/public that which is hidden, not to make a watered-down
life possible within unlivable conditions but as a necessary path to
making true life occur now. The politics of ``consciousness-raising'' is
the material process of overthrowing oppressive political institutions
at the only point they really exist (where they enter our bodies), by
treating each other honestly and never as instruments, thereby
generating irrevocably bonded yet autonomous agents and collectives
incapable of being consistent with status quo institutions. In my tiny
little corner of contemporary Western radical politics, this is exactly
what I've been doing with Mark and a number of others.

§

Something about all of the lovely tributes that has given me pause is
the tendency to see Mark with somewhat rose-tinted glasses. Don't get me
wrong, Mark was a first-class intellect, an excellent writer, and he
made quite an impact on a sizable audience. Many people knew Mark and
his work much better than I do, but from where I'm sitting I don't even
see Mark as primarily a writer. To me, Mark was an active revolutionary
first and foremost, he just happened to write a lot of things down. I
think this is really important because, how do you think anyone becomes
an important writer? It's certainly not by choosing to become an
important writer; it is by having some above-average source of interest
or energy toward certain questions and writing things down along the way
because you need to make sense of things as you go. Personally, I think
Mark was interested in how such a rotten set of institutions can
perpetuate themselves, and of course the question of how to overthrow
them. I see his writings as by-products of the much larger qualities,
attitudes and behaviors that made Mark the uniquely important figure he
was.

In a comradely way, I would even wonder if there is not something
possibly ideological in some of the glowing obituaries of Mark as a
writer. As if his obscure, independent k-punk blog became so valuable
and influential because of his way with words? I doubt that. And if you
want to grow up to be cool and valuable and influential then just start
an obscure, independent blog with good words? Maybe, but I think the
real reason Mark made a lasting contributon to late 20th century British
culture is because he fucking hated capitalism and it was killing him
and he actually dared to say so, and to explain how and why, and to
actively find others with whom he might take an honest shot at changing
everything. If that's the type of person you are, if that's how you
live, then anything you scribble on the back of a napkin is going to be
fascinating, inspiring, useful, and impressive. Not because you're a
good ``writer'' but almost the opposite, because you care so much more
about seeking liberation than being a successful ``writer'' that you
have the freedom and energy to do something real with words. This is a
crucial lesson for those interested in pursuing their own path of
radical cultural production, but it's one that tends to be erased in the
tropes our cultural industry uses to describe important writers.

No doubt I liked and admired Mark's writings, but I think Mark would
understand my wish to make clear that he was not some sort of super rare
genius talent. He wasn't: he was you. Of course he was smart, and a good
writer, but he was also weird and awkward and nervous, like you, like
me. I have met certain towering intellects whose mental function is in
fact probably something incomparable to what you and I have. Mark was
not that type, he was something far more dangerous. He would often say
interesting and brilliant things and also things I hardly understood or
did not agree with or did not find interesting. I've heard people call
him a great speaker, and he was certainly quite a speaker, but ``great
speaker'' risks a crucial misunderstanding. He was great fun to listen
to and talk with, but he was not a great speaker in the classic sense
most people associate with that phrase. He was often quite disorganized,
mentally cluttered, elliptical, stuttering, longwinded, and---if we are
being honest, and of course we are---sometimes downright
incomprehensible.

I remember at a Plan C Congress he gave a talk on some ideas from
Operaismo and I left the room with almost no idea what he was trying to
say. But the radical insight here is that \emph{that} can be more
politically powerful and sometimes even more fun and cool than ``great
speakers.'' This is exactly the political-psychological mechanics of
punk, where it is a lack of certain skills combined with a kind of
passionate carelessness that triggers real excitement and empowerment in
others, more so than mastery. So to call Mark a ``great speaker'' risks
the very same media-spectacle recuperation that pacified Punk. I'm
overjoyed to see Mark becoming a legend even sooner than I would've
predicted, given the remarkable outpouring of acclaim in the aftermath
of his death. But if the effect is to increase the perceived cognitive
or performative distance between the average reader and Mark, then that
would be unfortunate. What made Mark so interesting and powerful was
that he thought what he thought, and he said what he said, because he
wanted to, because he was irrepressibly moved to overthrow an
intolerable state of things. And he said what he said \emph{despite}
that he had all of the shortcomings and deficiencies of the average
person. To hear someone like Mark think all this radical shit, and make
all these crazy statements, was so politically electrifying exactly
because he was not super gifted and had to struggle against obvious
normal difficulties. But he didn't give a fuck, because he was a
revolutionary, and that could be you tomorrow, today.

Or consider what is probably his most famous work, \emph{Capitalist
Realism}. It's a totally cool little book that's fun to read and I think
it was really useful to a lot of people. But it's crucial to celebrate
it for the right reasons, and avoid those that distort Mark's unique
powers. It was not super original, certainly not systematic or
comprehensive, and it gave very little direction on what any of us
should do next. Mark wasn't a genius, he was an interested, passionate,
creative person on a search for something real, and that's so much more
revolutionary than mere genius. Again, his work was something you could
do, if only you could find the courage and energy to pay attention to
what really interests you, and write down what you think, for your
friends, precisely without really giving a fuck if its original or
systematic or impressive. This is the secret recipe of radical culture
that actually produces effects on people, and I'm pretty sure Mark would
not mind me reminding people of this.

From my view, I think Mark had a few key insights and I would summarize
them as follows. All of this is temporary and it's not supposed to be
like this, but if you look closely you can always find glimmers of life.
And it's necessary to find those glimmers of life and invest in them,
and if we all do this honestly and openly than we can and will find a
way to change everything. These insights are insights that many of us
have deep down inside, he just went after them as if it were a matter of
life and death, because it was a matter of life and death, just as it is
for us today, whether we feel like facing it now or later.

§

Mark's death is teaching me that our revolutionary moment today is so
much more real than I thought. Not an abstract potential, but something
that is already operating wherever radically true relationships are
being formed, if you only know how to pay attention, be honest with
yourself and others, and invest your energies wisely. The more you take
your attention and energies away from status quo fixations, and divert
them into those people genuinely attuned to liberation, then as the
dynamics of genuine bonding and belonging take hold, larger collectives
can be spun from the two, to the three, and so on. If Mark's readers
trust him as an authority on the political nature of depression, then we
should also trust him as an authority on the real and immediately
available road to revolutionary transformation that he and I and others
have been stumbling down together for the past few years.

There's nothing magical or sacred about Plan C, which is only one
particular group trying to figure these things out; it's about the
discoveries many people are making and are continuing to make,
discoveries which anyone can pursue in their own way and on their own
terms with anyone around them. This is not a vague appeal for everyone
to ``come together in love'' with everyone around them, not that at all:
it is an appeal to break away from all that is wrong and false with ony
those you can trust to make of yourselves whatever it is you need to
make of yourselves in order that life may occur together now. Not a
universal love, but a highly careful and discriminating love---which
might very well produce some enemies in the short term---based
rigorously only on those principles you honestly believe to produce real
dynamics of liberation, and an unflinching refusal of anything else. Not
a circle of people singing kumbaya, but a real uprising that honestly
feels like an uprising and which creates, almost out of thin air, the
very thing you have been seeking all along.

At least for me, this is how it's working. It's sad to say, but it
might've been Mark's death that has really driven this home to me once
and for all. During my last visit to the Plan C group in London, I had
the good fortune of spending some quality time with several of my
closest friends in that group. In a few moments, spread throughout my
visit, I had the distinct feeling that, with those people, I'm truly
embedded in a life or death struggle, but at the same time, in those
very moments, I felt fully 100\% alive and doing exactly what I was
supposed to be doing. You can't call it liberation or revolution
exactly, because no one is liberated until everyone is liberated, but it
was a really unique and overpowering quality or experience of life that
I used to think was something that would only come after the revolution,
as it were. In these moments I sincerely felt like it was already here,
or presently swelling like a wave, like it was actually happening in my
body, like we are really doing the only thing that revolution could
possibly be: our radically honest best, together. Interestingly, the
only time I can remember having this feeling was in the headiest days of
Occupy. The reason this is remarkable, and more evidence that indeed we
are onto something, is that feelings of revolutionary power are
supposedly short-term, fleeting, unsustainable rushes that only come
about in rare insurrectionary upsurges such as Occupy---but here I am
feeling them in a random pub with E, in another pub with N and A and J
and W, on the overground to Tottenham with S, with C and T and A and J
and A and S and everyone else at the same Misty Moon where I first met
Mark Fisher before a Plan C meeting in 2014. And I still feel them right
now, weeks later, even though to the naked eye ``nothing is happening.''

In a way that I wouldn't have said even two weeks ago, it really now
does seem to me that we are already doing it. I've never seen it so
clearly. I do not feel any hope for the future, which I firmly believe
is a conservative affect. What I have is an interpretation of where I am
and what is going on around me and who exactly are these different
people. And I have increasing reasons to believe that my interpretation
is true, while the socially dominant interpretation is false. What I
also have are concrete tools, reproducible tactics and techniques to
make energies flow inside our bodies, tactics I have discovered with my
revolutionary friends, whether we have fully realized it or not, tactics
that I can now creatively employ to remake every part of the world that
I touch. What's even more remarkable is a peculiar strategic assymetry
about these tactics: these are tools that only real revolutionaries can
learn, for the simple reason that today one must enter a revolutionary
attitude to even access certain basic human experiences prohibited by
what is currently called normalcy. Not least of these basic experiences
is the one I mentioned above regarding ``consciousness-raising,'' that
most primordial experience of being present with others for no ulterior
purpose whatsoever.

That simple and immediately available place of radical honesty and
being-unto-ourselves, easy as it sounds, is available only to
individuals and groups able to see that it is effectively barred to
normal humans adapted to the status quo. Also it is only through radical
relationships, attuned and bonded around the honest search for
liberation, that currently atomized individuals can gain the courage to
take the risks necessary for shooting down this path. When I speak of
risks I don't mean anything grand, I mean even just that blog post
you've been meaning to write but for some reason you're just vaguely
afraid to post. With honestly revolutionary friends, you stop caring
what the Big Other will think, and you say a little more, do a little
more, than you normally would---because you actually believe you're onto
something, as your friends are onto it also, and you might be crazy or
stupid but you can't all be crazy and stupid.

Last but not least, you begin to realize that even if everything fails
and everything goes wrong, nobody can really touch you, because the
truth is most people won't even know what you're talking about. At first
one's fear is always that people will respond negatively and punish you
for sticking your neck out, but as you learn to do so, buoyed by
revolutionary friends, you realize something at once more horrifying and
liberating: you are much more likely to be ignored or misunderstood,
possibly forever, than maligned and punished. If you're honest path
brings malice against you then you should count yourself lucky, for it
means you are certainly onto something. See the Vampire's Castle. Of
course you could also be veering toward evil, always a risk, but again
that's why you've invested so much into your revolutionary friends. They
will keep you honest without oppressing you.

And let me tell you one of the most beautiful things. If I haven't made
myself clear or you just don't understand what I'm talking about, I am
sorry about that but I also don't need to care or worry because I know
with certainty that at least a few of my comrades will. I'm able to know
this with certainty because the only reason I'm able to write this is
because of them, so almost by definition they will find themselves in
it. Radical political groups are often mocked for being self-referential
little spheres, but the only reason this is mock-worthy is because we
feel like we have to be accountable to something or someone else outside
of those circles. So the inside of those circles can feel sad and guilty
and lacking something. What exactly are they lacking, though? Nobody can
ever say. We feel like we need to do something more, or do something
bigger or better outside of ourselves, and we mock ourselves for being
tiny and self-referential only because we judge ourselves from the
perspective of some stranger in the big dead center who in fact is not
looking at us, and never looks at us. Ironically, the really perverse
thing about our little circles is that they are not radically circular
enough.

There's nothing wrong with a small group that makes time and space to
see nothing but itself. But the crucial condition for this to become
revolutionary, the condition which is so hard to meet, is that such a
circle must dare to make its own judgments about what is true and not
true, real and not real (not in the sense of one objective truth but in
the sense of diverse honesties or consciences), without apology and
without paranoia and with absolutely zero respect for the millions of
idiotic responses that might come from the massive dead center of
society. And then it must dare to really believe and live by those
judgments. The capacity to generate charmed circles is an extraordinary
political power. All that is necessary from there is to make that circle
expandable with a scalable membrane, not to self-loathe the inherently
circular nature of a shared world, constantly fearing that we are not
already enough for each other.

\subsection{Consciousness-raising and the bio-chemistry of revolutionary
politics}\label{consciousness-raising-and-the-bio-chemistry-of-revolutionary-politics}

Everyone asks: what is this odd dream we keep referring to as the social
strike? How can we speak of striking given that the strike is, as they
say, no longer a thing? I can already hear my activist and organizer
friends snickering, that I speak so dismissively of the beloved strike,
the proof they once had some power and will again. But indeed, activism
and organizing in general is really no longer a thing. I, and presumably
most other people, are bored by activists and organizers. Even those I
am closest to, even those I have the most relative faith in, I am
typically bored by them to the degree their perspective still basically
revolves around how to make activism and organizing ``work.'' For my
part, I am writing here not as an activist or an organizer, just one
person who opposes the current form of human existence and is dedicated
to collectively overhauling it before death takes me, as it did
yesterday and as it will later today. I begin this essay on personal
terms not because my own experience is especially important, but to
offer my own case as a kind of introspective ethnography. I could write
this as a sharp impersonal theoretical piece hoping to establish myself
as a cool and useful theorist---of course as a typically insecure and
alienated person I crave such recognitions---but that is precisely the
type of elementary dissembling that constitutes a Valuable Activist
Contribution at the cost of preventing honest and therefore real
revolutionary connections from emerging. In a society as dishonestly
sophisticated as ours, hopelessly useless honesty becomes revolutionary
method, as I will try to demonstrate here.

The problem is that whenever I start talking in terms such as life,
existence, and death---general terms, loose terms, honest terms, that
is, everyday speech about my life, my feelings, my deeper fears and
goals and what has always brought me to back to the revolutionary
position---it is always the most active of the activists and the most
organized of the organizers who are least able to understand. In many
cases, with sometimes laughable rapidity, it is assumed that I am simply
not educated in the radical activist catechism. If I give some
historical or theoretical references to convince them I know a thing or
two, I am typically pigeonholed into whatever niche of radical theory my
vocabulary or appearance seem to suggest. At the very best, the kind
ones will try to help me turn my positions into a proposal or some such
product that makes activists and organizers happy.

For a while, I took these exchanges seriously. I stepped back, I
listened to my elders, I learned, and I blamed myself: maybe I am just
being negative, maybe I am just obsessed with my own feelings, maybe I
am lazy or inarticulate or mumbling, maybe I just do not have anything
to contribute to revolutionary politics. Ever since the end of Occupy
back in the US, I have done my respectful best to contribute
meaningfully to currently existing struggles and projects. But each day
I fail to experience genuine revolutionary social transformation, I find
it harder and harder to justify investing myself in activist projects.
All the typical bourgeois excuses for resignation become increasingly
tempting, compared to the blind alley that almost all currently
available activism and organizing just honestly feels like.

From this experience, an insight emerges. If activist engagements
sometimes make me think about quitting altogether---if radical activist
circles cannot at a bare minimum retain the political energies of a
relatively privileged and educated person who is actively and purposely
trying to defect from status quo politics into dedicated, organized
revolutionary politics---then really the activist incapacity to
co-mingle and multiply with non-activist revolutionary energy must
itself be a primary problem. If there is an ``energy crisis'' today, it
operates much more at the level of interpersonal bio-chemistries than
fossil fuels, for only the former can explain the absurdity of the
latter: only a world-historical pacification of basic energies could
explain how many humans cannot even find it in themselves to resist
their own extinction.

I should be clear that I am not trying to contribute to currently
existing activism, in which case I would be easily pigeonholed into the
anti-activist activism tradition. I love my activist and organizer
comrades and friends, but whenever I frame my perspective as a simple
friendly offering, I am beginning to realize it is they who sometimes
honestly cannot decipher that I am actually raising a serious challenge
and intimating a fundamentally alternative model. So I am not trying to
constructively engage activism from within it, which has never achieved
anything; the critique of ``activism'' from within radical circles is
long-standing, but activism is too good at protecting itself. If it
seems like I am coming out swinging, it is only because I am
increasingly dumbstruck by how good activists are at not hearing
anything but their own language. I am not trying to distinguish myself
against anyone, itself a typical pathology of activist behavior which I
would rather deflate at its roots.

Almost all currently existing revolutionary politics is essentially
doomed to prevent revolutionary energy from appearing within its orbit.
The entire history of radical politics under capitalism is a tradition
in which activists primarily sell the commodity of optimism, in order to
earn social capital for their activist firm (in the form of members or
cachet). The traditional vessel for delivering customers the optimism
they now shop for is some practical strategy for obtaining collective
benefits. Long ago this was naturalised as just, on the grounds that one
honestly believed one's activist firm would deliver the collective
benefits to make it all worthwhile. The problem is that this treats
humans as objects to be organized, so that even if the collective
benefits are obtained, the humans who joined the activist firm are
farther from liberation than when they started. Activists duped them
into relinquishing their humanity for material benefits, and indeed they
helped dupe the activists into believing that such a basically
dehumanizing organization of relations and priorities could ever advance
liberation.\footnote{Here I feel like I should cite a long list of
  readings to establish my intellectual bona-fides and urge you to read
  them, but there is no reason to assume that more reading would make
  you more revolutionary. I don't even know you. If you want to, just
  Google shit or I would be very happy to email with you; if you don't
  want to learn more, then me telling you to read certain things will
  just make me seem more powerful than you and make revolutionary
  politics feel to you like work. Why radicals think it is revolutionary
  to recommend readings to people, I cannot understand for the life of
  me.}

Today, very few of us would ever actively lie or manipulate or mislead
for radical ends which we believe would justify such sinister means. But
yet, very few of us want to be radically, unlimitedly honest as an
independently worthwhile aspect of revolutionary transformation in its
own right. Who would want to proclaim that we are hopelessly alienated,
overworked, unhealthy, atomized, insecure, and simply have no idea how
collective liberation can be produced from where we currently find
ourselves? But yet, on the whole this is the case, and whenever these
facts appear they are neurotically repressed in truly symptomatic
fashion. Even in the most open organizing spaces, too much theoretical
interest in these facts is seen as immature, ill, digressive,
narcissistic, undeveloped, or at best tolerated as a therapeutic need of
the speaker (which we later need to figure out a way to strategically
translate into something practical). Revolutionaries remain unable to
see that such naked facts of our predicament are themselves of primary
interest, and that they are produced by orienting ourselves to each
other as objects to be strategized, not solved by such an orientation.
This is likely because it would not sell to simply say we are united by
our radical honesty in having no answers, and so beholden are we to this
logic that if we cannot imagine it selling then even we have a hard time
believing in it. And it would not satisfy the need of current activists
for whom activism is its own satisfaction as a platform for feeling
ethical and efficacious and optimistic.

This need of activists to feel that one is already a valuable member of
an adequately effective political project is one of the greatest enemies
to initiating liberating social contagions simply because it is false:
we are not really valuable members of any nearly adequate political
project, so long as we remain this incredibly far from true freedom and
equality, or anything that is even approaching its achievement. Most
people are ``apolitical'' because they know better than us not to be
duped by us. And within our own circles it is markedly
counter-revolutionary to try and translate honest despair into something
practical, almost by definition something falsely practical. Indeed it
is the activist addiction to optimistic practicality that enshrines us
in the despair of being nothing more than objects, because whenever
honesty breaks through the charade of false strategic shrewdness, the
obsessive desire to make something valuable out if it returns the
subject to an object, precisely when the subject is trying to break out
of value production, and precisely when and where it should be able
to.\footnote{Of course, I am not saying that rational strategic logic
  should be exorcized from all aspects of human life, or that it plays
  no part in the thought-processes which feed into revolutionary
  politics. I am only pointing to a deeper, longer-term rational utility
  to the sincere and complete qualitative unhinging from strategic
  calculation. In other words, if you insist on the strategic attitude,
  then think of my position as a meta-strategy for overcoming why
  current strategies aren't producing revolutions; mine is a
  meta-strategy of radically defusing the strategic fetish in order to
  make possible the actual interpersonal dynamism which ''strategic
  thinking'' suffocates before it can even leave the gate. In other
  words, I am not pushing for a cult of purely irrational, non-sensical
  attitudes or behaviors; I am pushing for a highly rational disarming
  of certain deeply automatic defense mechanisms, rooted in instrumental
  reason, which prevent the group processes that make resistant
  sub-populations dangerous enough to force change. These self-pacifying
  automatisms enter activism before we even think about what to
  contribute to radical political discussions because they are rooted in
  the deeply habitual presumption of seeing each other, including our
  own selves, as means to ends rather than ends in ourselves.}

Lo and behold, there is a formidable tradition of radical politics which
seizes directly on the massive, latent repository of honest
relationality repressed by institutionalized domination (and traditional
activism no less).\footnote{Some will say that honesty is not the issue,
  that it is not a matter of looking deeper into our own souls to see
  the world as it really is. On this point I find it interesting that as
  radical critics we agree that modern society is capable of the most
  incredible mystifications, and yet the possibility that there exist
  pervasive mystifications at the level of our own self-understandings
  is seen as implausible. I find this more odd than the idea that
  radical introspective inquiry may be a natural dimension of any real
  revolutionary politics. Based on everything we know about modern human
  beings (here one could cite figures from Heidegger to Arendt to
  Goffman), I would think looking deeper into our own interior lives and
  our deeply habituated ways of living and presenting ourself to others
  would at least be acknowledged as a worthwhile area of inquiry.} Its
chief modern political signpost is known as consciousness-raising. Most
popularly associated with 1970s socialist feminism, I would define
consciousness-raising as the dedicated practice of honest relationality
which is not reduced to some larger strategic necessity to which it
would need to justify itself.\footnote{When I refer to honesty, one
  should not imagine merely the conventional image of a noble
  introspector who confesses their lies. True honesty implies the entire
  thread that links the emotions and judgments of individual to their
  everyday perception of themselves and their surroundings, and
  ultimately to the choices on which they will choose (or not choose, in
  bad faith) to stake their lives. In other words, far from reducing the
  political tradition of consciousness-raising to mere introspection,
  defining consciousness-raising around a generalized political honesty
  is indeed to view it quite epically. Of course, this larger
  perspective on the radical politics of the personal is what 1970s
  socialist feminism knew perfectly well. While 1970s feminist
  consciousness-raising was interlaced with a strategism which played
  its own important roles, what I am arguing here is that, empirically,
  I believe it is the radical will to honesty that drives, and can even
  constitute its own, strategic actions---for it is only the will to
  honesty which can energize and bond a reality realer than bourgeois
  society, which is then concretely dangerous, powerful, and genuinely
  actionable.} Consciousness-raising is also one significant focus
within Plan C. And yet, even within Plan C, consciousness-raising seems
to function as a side project. At the recent Fast Forward Festival,
where one of the big questions revolved around advancing the concept of
the social strike, it is remarkable to me that consciousness-raising as
political practice did not seem to come up (at least not in the sessions
I attended), as a significant pointer toward what exactly a social
strike might entail. Given that the concept of social strike remains to
many unclear, it stands to reason that our inability to integrate
consciousness-raising as central to the idea of the social strike is not
just an overlooked connection but a deeply conservative blockage
symptomatic of the dubious, actionist optimism described above. It is
hard to resist the inference that consciousness-raising may be seen as a
kind of feminised supplement to real revolutionary politics, even within
Plan C, useful as therapy and solidarity but only to the degree that it
feeds into practical strategic value. This is comparable to how personal
honesty shared in activist meetings finds no home unless it can be
translated into something practical for the optimism market.

Thus a second insight dawns upon me: In fact, consciousness-raising is
social strike in situ. Any instance of consciousness-raising, in the
general sense I defined above, is literally a social strike against the
value production and extraction which take place interpersonally in
almost all human interactions under contemporary capitalism. At the same
time that it halts the production of status quo value, it produces
fundamentally new sociality which is not merely ``feeling.'' If you have
ever had the experience of transformative interpersonal honesty, you
know that its products are veritable material resources. Clarity, trust,
empathy, acceptance, confidence in oneself, and ultimately joy, which
from Spinoza to Deleuze has long been recognized as essentially the most
elemental capacity to move and be moved: these are not feelings, they
are weapons. Learning how to collectively produce joy is the only thing
that ``social movement'' can really mean, in the first instance,
especially given our world-historically severe alienation and anxiety.
Consciousness-raising is not a feminine respite from strategic
revolutionary action; it is the science of revolutionary action itself,
for only this tradition takes seriously the crucial question of how to
produce collective energies from separated alienations in a fashion
which is not immediately stolen through some ``larger'' but alien sieve
which is somehow always labeled ``productive.''

This is what the Institute for Precarious Consciousness has in mind when
they speak of constructing machines for fighting anxiety. Although even
they see consciousness-raising as ultimately subordinate to ``action.''
My argument is that this instrumental subordination of
consciousness-raising to some kind of ``action'' is precisely where and
how capitalism short-circuits our subjective desires and atomized
powers, re-wiring them back into the status quo despite our sincerest
commitments to making revolution. In Inventing the Future, Srnicek and
Williams provide some striking etymological evidence that their own will
to subordinate consciousness-raising to ``action'' may perhaps be more
anchored in an exploitative logic of market value than they care to
explore: they ask, critically, when consciousness-raising sequences are
supposed to ``pay off,'' lest the affective bonds produced by
consciousness-raising ``go to waste'' (pp.~7). What if the problem in
activist circles is not so much ``folk political'' tendencies with
inadequate strategic perspective but rather the neurotic-repressive will
to hyper-strategically translate every possible human energy within our
orbits to some kind of radical version of ``cash value,'' which
precludes the very possibility of our alienated subjective energies from
fusing into larger collective dynamics? The problem would not be that
``folk political'' tendencies lead to failed social movements, but that
instrumental reason as the psychological bedrock of a capitalist order
makes us squelch each others' diverse micro-movements before
macro-movements could even be imaginable (as we try to make something
valuable out of each other, to appear and feel valuable ourselves, to
write books and articles, including this one, etc.).

In this perspective, it is not that consciousness-raising is a kind of
emotional therapeutic support to primary strategic collective action
such as the strike. Rather, most of what pretends to be strategic
collective action, such as the strike, is a kind of consumption or
expenditure (usually wasted) of the militant psycho-physical energies
produced by the everyday strategies of radical consciousness
maintenance. Consciousness-raising is the psycho-physical (emotional,
cognitive, and even bio-chemical) social reproduction of resistant
collective action itself. Therefore, in my perspective, to join the
social strike simply means to orient one's life toward maximising the
free, militantly non-exploitable joy circulating among those beings who
constitute one's lifeworld, including especially oneself. And to do this
in explicit opposition to all of the currently existing social and
political institutions whose very existence is literally owed to the
fact that they long ago agreed to turn these energies over to the status
quo order. Of course, it is immediately obvious that such processes of
energy maintenance already happen everywhere, every day. The problem is
that such energy maintenance is rarely if ever lived as a larger project
of evacuating status quo value and increasing the commons, in such a
fashion that can accumulate as an overthrow of contemporary
institutions. Even in most radical circles, we might brag in private
about our daily refusals and subversions, but only before or after the
``meetings.''

In my perspective, ``organizing'' a social strike means figuring out how
to cultivate a contagion whereby already common tactics of
micro-liberation become explicitly anti-institutional, increasingly
bonded in alliance (if only in implicit reference), and simply more
frequent, more widespread, and increasingly ungovernable. We have to
reckon with the possibility that the subjects we wish to become, and the
types of human energies we are seeking to proliferate, would actually be
too joyous to even submit to the ways of thinking and types of
interaction currently associated with radical politics. This does not
mean folks should not have meetings, but in this perspective one becomes
aware of how the object of meetings might be radically different than
what is baked into the current psycho-physical dynamics which bring
folks to activist meetings in the first place. This is not a rejection
of strategic thinking as such and is more than the basic
``prefigurative'' position that we should model the world we want to
see. Rather I am presenting here a specific causal pathway to strategic
leverage of a revolutionary magnitude, but the pathway is through a
radical unhinging from instrumental rationality (i.e.~the strategic,
essentially exploitative mindset which pacifies our mutual encounters
before they even begin).

When asked what is the meaning of this idea we call the social strike, I
therefore would say that it is simply the qualitative and quantitative
expansion and intensification of consciousness-raising to the threshold
at which the collective store of energy we withdraw from status quo
value (reserved for our own non-instrumental enjoyment and flourishing)
outweighs the store of value held in reserve by the status quo. When I
speak of energy I am not referring to any kind of mystical or New Age
notions: all I have in mind is the process of literally making ourselves
and each other's immediate lives actually more valuable and desirable
than anything the institutional status quo can offer, until this becomes
publicly and zealously true for a non-trivial percentage of the
population.

It is not that ``we'' have to radicalize the ignorant and complacent
others, quite the contrary; activists have to educate and practice
ourselves out of our current, generally dishonest selves to simply
become ourselves honestly, as much as possible, not just in special
private spaces but increasingly in all spaces at all times, so that we
increase our life with others (the commons itself) in actual movements
of irresistible and uncontainable defiance. Because this would represent
the real acting out of new collective energies rather than the
performing of old energies we merely assume we are supposed to perform,
we would be producing a source of actual, lived power from which
strategic collective action might once again become possible. The spread
of this type of behavior and attitude in deeply held and publicly
political, interpersonal solidarity, beyond as small a threshold as 10\%
of the global population, would represent a fundamental threat to global
capitalism. I am not saying this would be easy, but it is at least a
concrete macro-vision with what social scientists would call
``micro-foundations,'' i.e.~a plausible account of how individual-level
dynamics could feasibly aggregate into the desired macro-level
phenomena. This is more than one can say about many implicit mental
models of revolutionary change dominant in activist circles, as
evidenced by the simple fact that people are not exactly flocking to the
revolutionary cause. Not yet. If educated progressives are so smart, why
is right-wing populism winning?

\subsection{Is it dangerous to engage seriously with radical right-wing
thinkers?}\label{is-it-dangerous-to-engage-seriously-with-radical-right-wing-thinkers}

First, I am generally skeptical that any intellectually radical thought
or speech that takes place within the wealthy liberal democracies today
has much effect on anyone, for better or worse. So it is hard to see how
intellectual or even social engagement with polar ideological opposites
(even if one thinks their positions are plainly wrong or violent) would
itself cause unique negative consequences of any kind. The genuinely
thinking members of the ultraleft and ultraright are so few and marginal
that it strikes me that this fear of ``dangerous associations'' is a
narcissistic delusion; it's a way of pretending that anyone gives a shit
about what we think or say. Nobody cares, and that's an essential part
of our predicament, that human beings today are quite fully pacified and
neutralized in their capacity to be moved.

It stands to reason that, if there is any way out of our collective
virtual jail cells of separation and alienation, it might have to
involve finding those who are most alien and frightening, to see if
perhaps they know something we don't. To see if, possibly, one reason
the revolution never comes is that it's always been posited on this
obsessive insistence that our evil enemies could not possibly know
something we do not know. Even if you believe someone is evil, you can
still wonder if they know something you don't. Clearly, collective
revolutionary liberation has not occurred yet, so almost by definition
there must be a large number of things ultraleftists remain
fundamentally incorrect about, right? I don't think it's such a crazy
idea to suppose that the most intelligent people on the right could be a
source of great insight, especially into the dumbest and most
ineffectual aspects of ourselves, aspects which we are possibly unable
to see as clearly as we need to. I understand this is a moral heresy on
the left, but what if this very structure of paranoid-neurotic moral
prohibitions is itself one of the most dangerous problems, and one we
are uniquely ill-equipped to see in ourselves, by ourselves? This seems
not only plausible but, in my view, increasingly likely to be true.

There is this fear that engaging with what one takes to be bad ideas
will function as legitimation of those ideas, and that this will then
spread them. There are two really hard problems with this view. First, I
think we have to wake up to the empirical reality that left-wing culture
has been collectively practicing this strategy of disengagement and
disavowal for the past several years and the effect has been a
\emph{flourishing} of radical right-wing perspectives today. So the
``distance and delegitimate'' strategy of dealing with one's ideological
opposites is arguably the most dangerous way to relate to those one
disagrees with (this is assuming one is correct and one's opponents are
wrong, that the others are in fact dangerous, which could itself be a
pathological arrogance also symptomatic of what's wrong with us). Hence
the second problem with this attitude is that it assumes one knows for
sure one is fully correct and one's extreme ideological opposites are
fully incorrect. I think this is a patently stupid presumption; most
people use political ideas as blankets to keep them warm, and that's
fine, but the reality is that no matter how much one believes what one
believes there is always the possibility that perhaps one has been wrong
all along and one's ideological opposites have been correct all along.
This may be very unsettling, but it appears to be an unavoidable
difficulty baked into reality itself, and no amount of militance can
overwrite such difficulties by force.

Finally, even if it is ``dangerous'' to liaison with thinkers of
seemingly evil thoughts, I would say the whole point of a radical or
revolutionary political position is to actively cultivate a
higher-than-average tolerance for danger. Is it not odd that people who
claim to believe in radical critique and revolutionary social change
speak of ``dangerousness'' as something to be avoided? Again, I see here
a symptom of the mystified mechanisms of our own pacification. The truth
is, I desire some danger. If only my thinking and communication
behaviors were dangerous! If there is any danger to dialogue with the
radical right, I think I would simply say that I am not afraid of it. I
trust in the capacities of human beings to distinguish true from false
in the long run, I trust that we will converge on the truth ultimately
but only if we, perhaps somewhat dangerously, are willing to consider
everything, especially that which we are most emotionally invested in
(for that is almost certainly where our errors will be found). And I
believe strongly that whatever the truth about human beings is, finding
it out cannot hurt us any more than all of the prevailing falsities. The
catch is that one has to at least entertain the possibility that what
one fears and loathes the most could potentially turn out to contain
some truth. If that's dangerous, then I would think it is precisely the
type of danger revolutionaries should be faulted for not seeking out
more eagerly.

\subsection{On turning left into
darkness}\label{on-turning-left-into-darkness}

A response to critics and a more positive elaboration of my current
thinking.

In the past week, many associates of mine on the radical left have
expressed grave concern about my recent cultural politics. If you
haven't been following, here's my best shot at a succinct, impartial
recap. I have been blogging about what it means to engage intellectually
with what I call the ``smart right.'' By that I mean people with
personal intellectual projects defending right-wing perspectives,
potentially even including some that appear horrifying and/or evil. I
use the word ``smart'' only to exclude from the mind two images that
have come to define ``conservatism'' in the left-wing imagination:
infantile and fundamentally disingenuous politicians, and then mindless,
racist armies of trolls. Specifically, for instance, I have expressed
interest in the writings of Nick Land and Curtis Yarvin (Moldbug); I
recently hosted a podcast with psychologist Diana Fleischman that
included discussion of controversial topics such as ``human
biodiversity'' (some say this is a euphemism for racism and some say
it's an empirical reality). Obviously not the usual talking points for a
left-wing intellectual, but to be clear nobody is accusing me of writing
or saying anything particularly impeachable. I did receive some very
thoughtful concerns, however, so one goal of this post is to clarify at
least one or two of the most fair and important criticisms I have
received. This is a caring thing to do, and I still believe deeply in
caring.

On the other hand, any culture of absolute kindness becomes a
conservative system of unspoken violence insofar as painful truths get
repressed and all participants become deformed over time. It is because
I genuinely love my friends on the left that I am stepping up to
publicly state, and seriously pursue the implications of, dozens of
difficult questions we have basically had an unspoken pact to not speak
about for perhaps decades now.

If you are one of my comrades on the left who is generally overexposed
to human docility or illness, I must also warn you, caringly, that you
might be alarmed or confused by what follows. Many of you are now
accustomed to a particular script: comrade is ``problematic,'' group
pulls moral alarm, comrade begs forgiveness and (even in the best of
cases, not to mention the horror shows), comrade dies a little on the
inside, group feels reassured comrade will do no harm, groups grows old
and gray wondering why they never changed the world. Well, I have seen
this script performed too many times to play along any longer; over the
past several years I think I have learned a thing or two about why our
groups don't change the world. One reason is that we punish our own for
grappling with questions we pretend to understand but are in fact to
fearful to seriously consider.

So at the same time this post will charitably respond to some left-wing
critiques of my project, in the same breath I am going to
unapolagetically push further outward on my perspective that so
horrifies many of you. I will no longer fight rearguard battles against
fearful and disingenuous people on the left who would rather condemn
something than admit they don't have the time to read and process it;
but neither am I here to cozy up with right-wing currents, as so many on
the left assume of anyone who starts really speaking up and speaking
out. I should like to become a \emph{worthy opponent} of the smart wings
of the new reaction, rather than merely pretend they are stupid; for I
consider it a great embarrassment that the revolutionary left has yet to
generate anything as genuinely interesting and creative as The Dark
Enlightenment or Unqualified Reservations. If so-called
left-acclerationism is our best response, then we're in deep trouble
(see below). Fortunately, I think we can do much, much better, but we
won't know until we try.

One of the key objections put forward by my more thoughtful critics from
the left is the following. They argue that it is ethically and/or
politically wrong to entertain a \emph{frame of debate} in which racist
implications appear likely. For example, my podcast with Diana is
ethically or politically bad because by even discussing biological
differences across groups, I am effectively increasing the perceived
legitimacy of notions that can and will be used to support racist ideas
or policies. I think this is a reasonable concern based on a plausible
model of culture. Yet after reflecting on this for several years, I
believe this idea is fatally mistaken in ways that have not yet been
fully grasped or written down anywhere (that I know of, anyway). Here is
a first, short attempt.

This idea that it is ethically or politically wrong to \emph{entertain a
certain frame of debate} is a fatal error in both the normative and
empirical sense of that term. First, on the normative level, the idea of
refusing to engage people with certain frames of reference dehumanizes
people who have no access to anything other than those frames of
reference. In short, this objection writes off large swaths of humanity
as inhuman. I believe that this monopoly on humanity claimed by educated
leftists is now, on net, a more violent and reactionary phenomenon than
any legitimacy that would be given to racism by even talking with a
proper racist (let alone decent people who merely have dicey or
controversial positions). What many on the left ignore is that today
large swaths of human beings are, through no fault of their own,
socialized into right-wing and often racist frames. There exists a large
number of people who are racist because they were sociologically doomed
from birth to be racist (e.g.~poor undeducated white kids in racist
families and geographies are statistically doomed to be racist). Their
humanity has been robbed from them (as it's increasingly robbed from
everyone).

It is my view that the revolutionary left is absolutely obligated to
treat such people \emph{as the humans they truly are} despite the
dehumanization they have been subjected to. When the ``humane'' leftist
says thou shall not engage with any racist ``framing'' of a
conversation, they are saying that large swaths of essentially innocent
people \emph{do not have the right to think, speak, or participate in
public life}, i.e.~this position coldly writes off the past and
continued dehumanization of literally millions of people. Leftists think
they are being radically humane, guarding the last line of defense
against the collapse of human equality, but the horrifying mistake
nobody is willing to reflect on is that this is actually saying ``keep
those filthy animals out of the little circle of humanity \emph{I} still
get to enjoy with my educated friends.''

The genuinely humane, revolutionary-emancipatory position in
contemporary culture is that we must dare to do the cognitively and
emotionally terrifying, and dangerous, work of extending whatever last
shreds of humanity we have, to everyone we possibly can. Therefore, the
truly humane, caring, revolutionary gambit today ethically requires us
to ``engage with racist frames.'' As a militant antifascist, I also
believe in drawing lines across which absolute refusal or physical
resistance becomes the correct move: to me, the clear line is if someone
is actively engaged in violence or directly inciting it. I would not
have a conversation with a neo-Nazi marching in my town throwing bottles
at immigrants; I would, with my community, physically remove them from
my town. All I am saying is that to draw this line of militant
non-engagement at the level of ``thinking and speaking with a racist
frame'' would require us to tell millions of people to go die in the
cesspool they were born into. We have been effectively doing that for
decades now, and not only does it fail, but it appears to
\emph{engender} or intensify novel mutations of racist politics (e.g.,
carefully non-explicit white ``identitarian'' movements, etc).

Continuing from the previous part, the second problem is as follows.
This notion that it can be wrong, \emph{a priori}, to consider certain
frames of reference is a grave error in the practical or strategic sense
as well, because to cast off so many peole as inhuman casts off all of
the humans we would need to change anything. It empirically dooms the
left to never achieve the fundamental transformations we claim to be
fighting for. If you listen to smart people on the right, they are
currently laughing their way to the end of humanity as the left
continues to push deeper and deeper into the mistakes we are actively
refusing to learn from. It is very difficult for the few revolutionary
leftists still alive to confront this, because it's genuinly so
vertiginous and horrifying that it really approaches what is cognitively
and emotionally unsurvivable for genuinely caring people: there are at
least some objective reasons to believe the human species may be
genuinely crossing the threshold at which exponentially increasing
technological efficiency makes the absolute end of humanity an objective
and irreversible empirical reality. I think it's debatable where we are
at in that process, but it seems undeniable this question is now
genuinely at stake and I simply don't see a single person on the
revolutionary left seriously considering this with the radical honesty
it requires.

If folks like Srnicek and Williams and the ``post-capitalism'' types are
the best the radical left has to offer on this front, I'm very sorry but
we're in serious trouble. No disrespect to those folks, they are all
very good and smart people. But that is exactly the problem. A really
profound problem nobody on the left wants to consider is that being
``good person'' imposes psychological constraints on your most basic
capacities to think and express yourself honestly. To understand this,
we need to take a little historical detour.

Recall that capitalist society only emerged and grew on \emph{hypocrisy}
as the standard mode for cognitively and emotionally managing the
necessity of having to brutally exploit each other to survive. This
hypocrisy is what the word ``bourgeois'' means, and it is nothing less
than the naturalized lifestyle of everyone who qualifies as a ``good
person'' in modernity. Because living as a human being under capitalism
requires hypocrisy, being empirically correct about what is happening
and how the world functions (science) as well as interpersonally
adequate to each other (what is called ``caring,'' or saying/doing what
helps specific other people in specific moments) are mutually exclusive
to a substantial degree. The psychologist Jonathan Haidt has shown with
several years of research that people who identify with the political
left are disproportionately interested in ``care'' as a value;
conservatives have a more multi-dimensional ``palette'' of moral
foundations). To be clear, I am in fact deeply interested in the value
of care, which is one reason I find myself sociologically on the
left-wing of political culture. The unique challenge I don't see anybody
on the radical left seriously confronting is how our committment to care
comes with the cost of certain systematic errors we happily ignore by
dishonestly repeating over and over that we ignore them because we
``care.'' The issue here is that, it is programmed into the nature of a
capitalist bourgeois society that to pursue unlimited ``care'' means
that you objectively do not care about changing reality, insofar as
changing something as complex as ``society'' requires extremely
sophisticated empirical rigor, deeply at odds with the care we also need
to exercise in order to cooperatively change things together as diverse
human beings.

(An aside. The first and most stupendous person to see all of this in
the early stages of capitalist modernity, who so clearly saw the doomed
destiny of any society organized on hypocrisy, that he preferred to
sacrifice his public ``goodness'' to produce monuments of honesty so
outrageous he hoped they would raze the hypocritical order altogether,
was, of course, Rousseau. Now, Rousseau did not squash the rise of
bourgeois hypocrisy, but he had demonstrable effects in generating the
modern revolutionary left tradition as we know it, from the French
Revolution to Fanon and beyond. There are many good critiques of
Rousseau, but if there is one example of how a sincere \emph{individual}
can craft a life that contributes to genuinely collective,
world-historical waves of revolutionary political change, it is surely
Rousseau.)

In my view, this tradeoff between being correct about how the world
works and caring for each other enough that we can cooperatively change
it in the direction of peace and abundance for all---this is perhaps
\emph{the} most vexing and urgent puzzle for a genuine revolutionary
left today. Yet remarkably I am not aware of a single person genuinely
risking themselves on solving it, so I'm going to try. At present I am
working on understanding the mechanisms whereby such an important
problem has somehow been so stubbornly invisible to so many of us for so
long. My wager is that we if we can truly understand the mechanisms of
our own blindness, we will find pathways to the holy grail of the
revolutionary left tradition: the flourishing of all human beings in
peace and abundance, immediately, without recourse to all of the
right-wing solutions that get raised in direct response to the left's
willful neglect of exactly this impasse.

It is because of this tradeoff between being correct and caring that I
have recently become interested in what I have been referring to as the
``smart'' right-wing. Many people are concerned that my recent interest
in intelligence means that I've become an IQ elitist or something. On
the contrary, I am keenly suspicious of the politics of high-IQ
subcultures, precisely because I know there is a trade-off between being
correct and caring. Because we care about each other, there are certain
things we refuse to see or else refuse to tell each other about what is
really true. That's fine, and perhaps a hard constraint of the types of
beings we are on the radical left. But ``smart'' far-right people, who
do not give a fuck about how people feel, \emph{they} might just be the
only ones \emph{capable} of telling us those truths we need to process
if we are ever going to have a sufficent command on reality to generate
the systemic transformations we believe in. But at the same time, I am
highly skeptical that the evacuation of care is a viable political
project, because warmth is a condition of life for we creatures who
require the sun to live, we creatures who are literally composed of a
once-exploded star. I think right-accelerationists are wagering on the
possibility that, if technologically super-charged hypercapitalism is
\emph{understood correctly} (hence the call to minimize care), that is
objectively the most likely path for the possibility of surviving,
perhaps into the becoming of something post-human.

For instance, a remarkable feature of Nick Land's current writing is his
obsession with coldness; I have never read anyone who so conscientiously
endorses the absolute evacuation of care as a political project. Many on
the left find this so evil they are resolutely insisting that if one so
much as speaks his name without any positive adjective in the same
sentence, \emph{that} very act is enough to force the speaker out of the
publicly defined circle of ``good humans'' into that outside zone of
cast-off inhumanity (consider that Land's handle is \emph{Outsideness}),
via the same intellectual-social process I described above. If we
self-servingly cast off human beings as if they are sub-human, we cannot
then feign surprise and indignation if they say, ``OK then! I'll go off
to become one with the superintelligent eugenically produced cyborg
overloads you'll be enslaved by in a couple of generations and I will
laugh my ass off all the way to the singularity!'' That's the vibe I get
when I browse Nick Land's ongoing work, and when I look at the objective
reality of runaway global finance and the tech sector, it does not seem
implausible that something like this could potentially be underway. Of
course I find that horrifying, which is why I am calling absolute
bullshit on the people who say that it's ``too evil to engage.'' I think
it's too alarming not to engage.

The more evil you think someone is, the greater should be your concern
to ensure there is not the slightest chance they understand something
better than you. If they are so evil, and they understand even one tiny
thing you don't, perhaps they are off using that edge in knowledge to
engineer you out of existence. This suggests to me that when people say,
``intellectual engagement with person X is prohibited,'' what they are
actually saying is ``we are so afraid they might be part of the
superintelligent cyborg army coming to enslave us that, even if they are
literally preparing to, we do not want to know about it, \emph{even if
there is a chance that we can still stop them!}'' And this is where I
get off the train to nowhere, for this is where moderate respectable
leftism (including most currently existing ``radical'' variants)
converges with the most insiduous and cowardly conservatism. If there is
some chance that hyperintelligent cyborgs are preparing to overtake
humanity once and for all, because there is some chance that for
generations now they have been operating on a model of the world we made
it our pact to never consider, then I'm going to take a real look. Not
everyone has to be comfortable doing so themselves, but at this point I
think that any honest, decent, thinking being on the radical left will
\emph{at least allow me to try.}

I believe that currently, a dirty little secret on the the left is that
for some people, the ``left'' is an agreement to protect each other's
right to look away from the most horrifying and potentially tragic
realities of planetary life today, to (implicitly) secure amongst
ourselves the last bits of interpersonal warmth available on the planet,
agreeing to \emph{allow} the rest of humanity's descent into
irreversible coldness. It helps to explain why, if you even approach
these issues with the slightest indication of analytical coldness, you
have to be ejected from the warmth cartel, for ejecting such existential
threats is a condition of its survival. But I believe it has always been
the vocation of the revolutionary left, properly understood, to risk its
own survival on deploying just enough analytical coldness to engineer
the unique machine that would take as an input the left's unique
material resource (warmth or energy via care) and produce as an output
non-linear, systemic dynamics the ultimate equilibrium state of which
would be peace and abundance for all. What that machine looks like is
the question, and this is only a formal statement to illustrate the
revolutionary left position today as an engineering problem. There are
many reasons that have been adduced as to why such a machine cannot
exist, and I do not pretend to offer responses to them here. I am only
suggesting that any revolutionary left today, worthy of the name, would
need to ``solve for X,'' as it were. The point of the engineering
metaphor is not that everybody in the revolutionary movement will need
to be an engineer, not at all; the point is only to show that any
left-revolutionary project, to succeed, will have to solve this
engineering problem.

What does this mean for revolutionary politics, in plain conversational
terms? By putting all of our eggs in the basket of care and kindness,
the radical left is now suffering from an engineering crisis it does not
have enough engineers to \emph{even notice}. In short, making revolution
is a complex practical problem we are not solving because we are now
generations deep in a long-term strategy of prohibiting people who are
good at high-level problem solving but bad at being polite. Not to
mention people who are good at creative and social openness, but bad at
obeying rules. Thinkers of the respectable-radical left, people such as
Paul Mason or Srnicek and Williams are selling a hope of technological
super-abundance, but they are too sweet to tell any of their left
comrades that all of the people you would need to \emph{actually
produce} that super-abundance are off building hyper-exploitative
super-capitalism in part because they once went to an activist meeting
\emph{and everyone treated them like fascists.} To bring this back to
anti-capitalist basics, the reason left post-capitalist thinkers don't
reflect much on such little problems as this one is because selling
books is as mutually exclusive with truth-telling, in the short run, as
is being a ``nice person.'' Hence the need for a fundamentally
anti-bourgeois revolutionary intellectual culture cold enough to seek
all of the darkest truths, but still warm enough not to betray the
calling of solidarity. I'm not saying the left should start worshipping
cold analytical power; all I'm saying is that if we genuinely believe in
the necessity of changing the world, a revolutionary culture would have
to be at least minimally hospitable to a minimal number of people who
have knowledge of how complex things work and how they break, and people
with the traits and inclinations to maneuver among diverse others. Both
types of people are effectively prohibited from those who currently
define radical progressive politics. Contemporary radical left culture
is now so fully doubled-down on the wager of kindness over intelligence
and creativity, that I am afraid it is almost vacuum-sealed against
learning why it might be on the verge of extinction. I am writing this,
and will continue writing to this effect, on the last-ditch possibility
there exist other people out there, somewhere, who can see in this
something more important than a moral offense.

\subsection{Two cultures of radical
politics}\label{two-cultures-of-radical-politics}

Why morality preserves the status quo and science is the key to
overthrowing it.

While many people on the left still pretend that ``free speech'' and
``political correctness'' are fake right-wing concepts, a number of us
are beginning to realize the profound mistake of dismissive moral
posturing. Get your popcorn ready now, because it's going to be a
fascinating mess as more and more people on the left begin to realize
that the cultural politics of policing moral symbols has been fully
exhausted and defeated.

The collective-emancipatory gains of genuine truth-seeking are now so
massive compared to the rapidly diminishing marginal returns to the
moral model, that there is no reason to spend much effort trying to
convince the remaining moralists. First, If I am right that the
truth-seeking model is better, then it will win because it is better,
whether I convince anyone in a blog post or not. Second, I am practicing
it, so if I am right then by simply thinking and doing what I am
thinking and doing, you'll see how it works in practice. If I'm wrong,
my ideas will fizzle out and I'll go away. In any event, what I would
like to do here is simply unpack some of the notions I have been
referring to in not-fully-explained shorthand. To begin, what do I mean
when I refer to the ``moral model'' and the ``truth-seeking'' model?

\textbf{The moral model vs.~the truth-seeking model of radical politics}

By ``left moralism'' or any of the other cognate phrases I sometimes use
to this effect, I am referring to the model of political activism that
seeks to change society by enforcing moral prohibitions. I think this is
far and away the most widely held mental model of how progressive social
change can and should be achieved, in moderate as well as radical
circles. Make a list of words and ideas and types of behavior that are
good, and try to get people to identify with them, talk about them, and
go to meetings around them. Sometimes these words seem very concrete and
action-oriented (such as ``strike'') but nonetheless, if you observe
like an anthropologist would, you find that an overwhelming portion of
the energy is organized around identification with certain words and
ideas believed to be in some sense normatively good or desirable. Always
curiously lacking is impartial assessment of effects. Also, make a list
of words and ideas and types of behavior that are bad and politics means
discouraging these by whatever means necessary. Radical politics means
\emph{really strongly discouraging} these things. An important feature
of this model is that what, exactly, should be on the list of bad things
is a question that is not in principle open to question or debate. It is
a characteristic of the moralist model that questioning its basic
premises is itself one of the bad things to be discouraged; ``good
politics'' means granting the goodness of the list and it's enforcement,
simply because that's the moral thing to do. In practice, today, what's
actually on the list of bad things is generally determined through
reverse dominance hierarchy in which deference is given to the most
institutionally dominated individuals and groups you happen to be
around. To be clear, I don't think this is a totally unreasonable model.
It kind of makes sense: society oppresses people unequally so give some
priority to oppressed people in defining what is bad and everyone try to
stop the occurrence of those bad things from happening. Not necessarily
perfect but fair enough.

I use the phrase ``truth-seeking'' as an informal and intuitive name for
what I could just as well call ``scientific method.'' The problem with
``scientific method'' is that for a lot of people this will sound too
grandiose for thinking and acting around everyday cultural questions.
Not to mention a lot of people think ``scientific method'' applied to
social questions is impossible or harmful to begin with (it's not, it's
just harder to apply to social questions than to something like physical
objects). But most people agree that our ideas about the world around us
can be more or less accurate, more or less consistent with how things
actually work outside of us, and most people can admit they have an
inner sense of when their ideas are proven true by reality (something
works as you expect it to), and when their ideas are proven false by
reality (something you are doing produces unexpected, undesirable
outcomes).

So when I talk about ``truth-seeking,'' all I mean is informally but
seriously subjecting all of one's beliefs, opinions, and mental models
of the world to the basic guidelines of scientific method in an everday,
intuitive fashion. Basically: everything you think is just a theory, and
everything you observe at all times is data you compare to what your
theory would have predicted; you need to actively consider all plausible
alternative theories and you update your mental models of the world
accordingly. You can, and should, have unique background experiences and
feelings and creative quirks; scientific method in no way discourages or
disqualifies any of that, as it is popular for naïve humanists to
suggest. Indeed, truth-seeking is actually the only way to remain loyal
to your unique experiences and quirks: the scientific method provides
the key for translating your unique data into power over your unique
environment, by subjecting your thoughts to objective rules that are
guaranteed to give you the best possible command of your unique
situation. So this isn't just an academic protocol; it's the only way to
live a basically honest and mature life, and I would argue it's a basic
pre-requisite for anyone who would hope to contribute to the elimination
of oppression by complex social structures.

So the ``truth-seeking'' model of radical politics is fundamentally
opposed to the moral model. The moral model says to begin with what is
currently defined as morally bad (typically through reverse dominance
hierarchies), and devote yourself to discouraging and generally reducing
the prevalence of those things. The moral model requires specifically
that nobody question the fundamental goodness of that model, or the
wisdom of certain items being placed on the list of prohibitions,
because the whole strategy is based precisely on forcing conformity to
Goodness. The truth-seeking model's only rule is that you must be honest
about your data and how you're making inferences from that data, but
otherwise everyone should just do their best trying to understand how
oppressive structures function and how to think/speak/act with others in
the precise ways that will predictably overthrow those structures in
favor of equality and liberation. The moral model's final endgame is a
world in which all badness goes away through mass conformity with moral
criteria. The truth-seeking model's final endgame is, through diverse
and totally free experimentation, we collectively unlock our true
functional relationships to oppressive social machinery while immanently
rewiring them into correctly-functioning liberation machines.

\textbf{Why the moral model will not go down without a war}

The reason the opening of a free-speech cleavage on the left is going to
be really messy is that a large number of people have so long schooled
themselves in the cultural politics of moralism, and have for so long
avoided the very different protocols of truth-seeking (i.e.~scientific
method), that such a paradigm shift will understandably be experienced
as a mortal threat to their identity. And we know that human beings will
sooner go to war than reasonably reflect on anything that threatens
fundamental dimensions of their identity. People have staked years of
effort and many of their social relationships on a model that is
suddenly obsolete, so it's reasonable for such people to be confused and
fearful about their place in the future of radical politics, let alone
society. Fortunately, scientific method has an extraordinary egalitarian
feature that goes woefully under-celebrated in radical circles: it's
equally demonstrable (ultimately) to anyone who is willing to work at
understanding it.

The whole politics of left-moralism is actively anti-egalitarian because
it's logic is not readily and equally available to all interested
parties. There are many social and economic factors that make access to
scientific method unequally distributed, of course, but it has the
uniquely egalitarian-emancipatory feature of at least being intelligible
and employable by all who can find their way to it. The protocols of the
left-moral model are not only beset by the same basic problems of
unequal access (this is why educational privilege is curiously the
single non-demonized privilege in left-moral culture), but the protocols
of how to think and act politically on the left-moral model are not
available to all \emph{in principle.} They are unequally accessible
\emph{by definition,} so even if they start out noble and true, there is
no way for large groups of humans to hold each other accountable to them
in a fashion equally consistent with their truth. The magical techniques
of being an ideal ally in the moral war---in which, one day certain
words are declared good and the next day they are designated
impermissible, according to a logic that does not exist out of the
declarations of those groups and individuals who happen to be at the top
of constantly shifting reverse dominance hierarchies---is therefore
inegalitarian in principle. This is not to cry woe for the exclusion of
white men from power (as will be the immediate rejoinder to my point
here), it is to cry woe for anyone anywhere who might like to enter
revolutionary movements for liberation from diverse starting points. The
left-moral model is inherently illegible for anyone who is not able to
go through narrow, fickle, local person-driven power dynamics to receive
the day's edicts on what is good and bad. Scientific method, while beset
by problems of unequal access as with everything under capitalism, at
least has the egalitarian virtue of being written down, basically
unchanging, and citable to all.

\textbf{Multiple equilibria}

I think a lot of smart and genuinely good people on the left operate on
this model simply because, as a really-existing cultural structure, it
can always inflict very real punishments they are not personally able to
risk at the moment (ostracism) and it really delivers rewards they are
not personally able to forego at the moment (social stability, standing
and status in the in-group, efficacy, purity, etc.)

But the whole point of being a \emph{radical} or \emph{revolutionary} is
to actively cultivate a higher tolerance for social punishment than
bourgeois normies, and less reliance on the everyday psychological
payoffs that bourgeois normies require to make their sad lives livable.

The revolutionary life, the life that genuinely risks itself in the name
of what it believes, operates on a totally different equilibrium.
Through cultivated attitude and iterating behavioral practice, we push
our social punishment tolerance to the human maximum (but no more), our
reliance on disingenous bourgeois psychological tricks to the human
minimum (but no less), but we set our truth-seeking and
truth-speaking/behaving high enough that it becomes a unique and
inviolable source of two key resources. First, it provides
motivation/energy replacing that which is lost by foregoing the
convential bourgeois channels, because any genuine process of
truth-seeking is by definition interesting, inspiring, and endless.
Second, it provides \emph{actual power}, for oneself and for whoever you
wish to share it with, insofar as increasing your understanding over the
social average unlocks concrete pathways to change the world around you
\emph{despite} that most people are content to leave the world as it is.
Read the biography of any well-known revolutionary in history (anyone
whose life itself participated in world-historical effects), whether it
be a political revolutionary or creative/cultural revolutionary, and you
will find they are not just different or more extreme than their
contemporaries. You will find they organized their life on this
fundamentally different equilibrium, a qualitatively different
organization of energy inputs and ouputs, which provide the sustainable
bio-chemical basis necessary for producing systemically transformative
truths despite extreme social punishment and very little bourgeois
subjectivity-maintenance.

The left-moralist model will protest the new school loudly and
insistently until one day you just don't hear from it anymore. This day
is probably much sooner than most people think. Very soon the whole
fashion of generalized moral condemnation will be so fully outed as an
intellectually disingenuous and practically conservative tendency, that
everyone will soon be pretending they never engaged in that embarrasing
old fad. And the new cool kids on the block will be all those who are
currently risking themselves on truth-seeking, those who were willing to
take a little bit of heat from sad moralists in favor of seeking
\emph{what really works} for producing large-scale liberation dynamics.
The reason I know this is not because I'm special; quite the contrary,
it is because some version of this pattern characterizes all epochal
transformations. A scientific outlook makes you larger by making you
smaller, for it allows you to find a humble but real role in a set of
infinitely larger objective processes.

\subsection{Capitalism is an instance not an
essence}\label{capitalism-is-an-instance-not-an-essence}

Reading Deleuze and Guattari, cosmic selection bias, and the room for
revolutionary praxis within unconditional acceleration.

Nick Land sometimes reads Deleuze and Guattari as if all the mechanisms
theorized by them are mechanisms unique to capitalism. I'm going to
argue that this is one of the many specific points in which the
accelerationist question remains far from settled, and in which the
foreclosing of revolutionary praxis strikes me as over-hasty.

To Land, \href{http://www.xenosystems.net/re-accelerationism/}{D\&G's
concepts are not just theoretically applicable to features of capitalism
but strictly synonymous with them}:

\begin{quote}
The D\&G model of capitalism is not dialectical, but cybernetic, defined
by a positive~coupling of commercialization (``decoding'') and
industrialization (``Deterritorialization''), intrinsically tending to
an extreme (or ``absolute limit''). Capitalism is the singular
historical installation of a social machine based upon cybernetic
escalation (positive feedback), reproducing itself only incidentally, as
an accident of continuous socio-industrial revolution. Nothing brought
to bear against capitalism can compare to the intrinsic antagonism it
directs towards its own actuality, as it speeds out of itself, hurtling
to the end already operative `within' it. (Of course, this is
madness.)''
\end{quote}

This is reasonable enough because capitalism certainly provides the most
stunning and historically consequential examples of these mechanisms.
But this is effectively a form of selection bias. This thing we call
capitalism is only the contingent world-historical catastrophe that has
made us conscious of these mechanisms through our feeling the violent
long-run effects of them having over-taken us.

The whole point of D\&G's project, in my view, is to identify very
general mechanisms; such that they can serviceably explain the
perpetuation of systemic oppressions but also serve as actionable maps
for spinning new, non-linear systemic dynamics (world-historical
transitions) from the most micro-scopic mechanisms. If ``decoding''
meant ``commercialization,'' why are their texts otherwise quite clearly
anti-capitalist? In other words, while I think these readings of D\&G
are often quite brilliant and productive, the current frontiers of
accelerationism have something of a problem around ``face validity.''

I think the biographical evidence makes it very hard to fathom that D or
G intended any kind of passive capitulationism, and their works are a
brilliant catalogue of calls to activity. Their writings are filled with
injunctions such as, ``Always follow the rhizome by rupture; lengthen,
prolong, and relay the line of flight,'' etc. Am I really to imagine
that all of these lines are trying to tell me that I should start a
business? I am not dismissing the provocative capitalist reading of
D\&G, I am only pointing out the obvious (which is surprisingly glossed
over by the current frontiers of accelerationsim): D\&G's call to
accelerate seemed pretty clearly to be part of a larger vision in which
any interested party could learn how to accelerate into liberation from
the inertia of systemic oppression; that the other side is more
desirable, and that we might even find each other there together.

I am not saying that the passivism or ``horrorism'' of Landian or
unconditional accelerationism (i.e., there's basically nothing for us to
do) is not possibly the correct, final conclusion that D\&G were simply
incapable of drawing; it is only to say that, insofar as accelerationism
is premised on D\&G, passivist interpretations should explain why D\&G
spent so much effort delineating all of those general mechanisms in a
general way, with so many inspiriting exhortations, if not to use them
for liberatory ends.

Many want to conclude that the call to accelerate forecloses
revolutionary praxis, but one of the seminal projects underwriting the
accelerationist turn, that of D\&G, is filled with micro-models of how
to produce macro-phenomena. Accelerating, in my view, means actively
pursuing these threads, only faster. I do not see any strong reasons
why, in D\&G, the call to accelerate must necessarily imply a sharp exit
from the project of fundamentally changing the world through
revolutionary action---unless you define D\&G's concepts as essentially
beginning and ending with capitalism. And I see no reason for doing
that, other than a selection bias in which we are overweighting the
reality of capitalism simply because it's the phenomenon through which
we first saw certain general mechanisms to work on planetary scale.

\subsection{Activism is a capitalist virus from the future, honesty is
stage-one cybernetic
communism}\label{activism-is-a-capitalist-virus-from-the-future-honesty-is-stage-one-cybernetic-communism}

Capitalism is horrible because it disposes humans as garbage, but the
motor of that garbage disposal is only the rabid human insistence that
we are better than garbage.

As knowledge has become so specialized, many people would rather ignore
or criticize highly evolved specialist discourses rather than admit they
simply cannot understand them. This allows specialists to emerge as
genuine masters who come to dominate others through this particular
dimension of objective superiority, whereas honest admissions of
ignorance and uncertainty would be an empirically and normatively
sophisticated basis for unifying the large mass of normal people into a
living project transcending capitalism (i.e.~mass movement). The crucial
political insight here is that self-aware stupidity is far more
scientifically correct and powerful than misleading performances of
intelligence. The overwhelming majority of public performances of
intelligence understate the significant uncertainty around even most
true claims, as this brings material and psychological rewards, and
little cost to the individual, especially as social media allow us to
mute opposition. But when we all pretend we're equally up to speed on
the objective dynamics characterizing contemporary society, this
simulation enforces and obfuscates the unfair distribution of
intellectual legitimacy based on cultural capital.

We either attune to the disinterested objective truths of being or else
the vicissitudes of capital attune us to an infinite future of hell.
Insist upon the ultimate human equality of our ultimately shared, cosmic
worthlessness (you are welcome to call this dignity, but you'd have to
believe in god). Otherwise, feigning equality where it does not exist
will always be a servomechanism of capital. It is a tactic of cultural
capital accumulation through brute force, which is not excused by its
being also weak and unsuccessful. The reality we have to process is that
the hierarchy of intellectual command today is not a gradually
distributed pyramid (an image that incorrectly flatters most people by
suggesting they are only a little below the top while also being above
many others). It is better visualized as one small group in a hot-air
balloon, with computers drilled into their skulls, drifting away from
almost everybody existing, all of whom are equally lost, confused, and
helpless. As I will try to show, this is not sad but rather an ecstatic
realization.

An activist who sounds convincing in a meeting might be 90\% wrong about
how the world works, but they are convincing simply because within their
milieu they have an above average stock of cultural capital (verbal
fluidity, education, seniority, etc.). The problem is not that this is
``unfair'' or ``counterproductive'' (common euphemisms), the problem is
far more drastic: in a frighteningly literal way, that activist is a
flesh-robot programmed and installed by capital, to serve the overall
stability and growth of a system in which capital is the only possible
adjudicator. By under-reporting our honest ignorance and uncertainty, we
misleadingly command from others a kind of pathetic, sterile respect
rooted in little more than their own comparatively worse illnesses,
whereas an honest reporting of our own helpless stupidity is generative
of energies for collective search (``most people are as stupid as I am,
so my chance of figuring out what to do is as good as anyone else's'');
sincere irreverence and non-conformity leading to the breakdown of
bourgeois repression (``all these people who want me to be a normal
servomechanism of capital are dumb and powerless''); an increase in
risk-tolerance through a decrease in false hope (``I used to be cautious
because I thought I had a chance of surviving, but now that I see none
of us will survive at present, I might as well try to do something I
find interesting, which, ironically, makes me feel like maybe there is a
chance\ldots{}).'' If we all admitted that, compared to specialists,
most people are equal in their absolute incompetence, we might just be
able to do something they can't.

Remember those human-computers I asked you to imagine floating off into
outer space, leaving us behind? If such cyborg intelligences designed
the optimal virus to ensure the spread of capitalism, and then released
it upon the earth, it would have to look eerily like the modern
left-wing project: an exploitative drive to symbolically out-exploit the
dominant exploiters, which convincingly presents itself as opposition to
exploitation, not only spreading the culture of exploitation but
super-charging it, as the competitive proliferation of obfuscation
reaches a density that no human being could possibly penetrate in one
lifetime. Not to mention that the already dominant exploiters also
possess economies of scale in the production of obfuscation, dooming the
left to lose even at the level of moral one-upmanship. It is the left
that initiates an arms race of increasing stupidity, on top of the arms
race of exploitation, without in any way decreasing the spread or
intensity of exploitation, indeed pushing it deeper and deeper into
instinctual intersubjective dynamics (the soul). If capitalism is a
series of injuries inflicted on the weak, the history of anti-capitalism
hitherto is a series of insults added to each injury, sold to the
injured as preventative medicine.

To speak of left-wing activists as possessed by a capitalist virus sent
here by a super-intelligence from the future might sound like science
fiction, but hasn't the average perspective always rejected radical
political theory as science fiction? That this diagnosis will sound to
many like science fiction becomes objective data that it might just be
correct, an authentic realization about how the world works and what is
happening, beyond what is conceivable within the ideological boundaries
of what is defined as reasonable. I believe the inevitable next move of
any coherent revolutionary anti-capitalism will be to accept this as an
objective characterization of our history on this planet, and to update
our beliefs and activities accordingly. Ultimately, only time will tell.

At first, these realizations may be depressing, but depressingly true
realizations are the price of entry to the ecstasy of moving from lost
to found. When the lost find themselves in the radical equality of
honest and absolute helplessness, my hypothesis is that we will be
uniquely capable of moves that are psychologically and sociologically
prohibited to those who currently command their specialized sectors of
the global cybernetic apparatus. Ultimately, they are the most enslaved
by this system, that's why the system selected and promoted them. We are
those who were unfit to be slaves of this system (anyone silly enough to
still be reading this blog post on the internet is, by definition,
included in this ``we''), that's why we are being left to rot at the
bottom of its garbage disposal. But one of the primary reasons this is
so oppressive is simply because we are obsessed with the excruciating
and exhausting work of pretending the situation is otherwise. For those
willing to acknowledge where they really are, new and unexpected winds
begin to blow, the room to maneuver suddenly appears tremendous, and
communication once again becomes possible, as if for the first time.

It is only the fearful and defensive will to decelerate that makes
acceleration so brutal and insufferable; genuine submission to
unconditional acceleration appears the only pathway to a sensible and
durable calm, if only the white noise of unbiased chaos. Nothing about
this prohibits creative, collective emancipatory projects to establish
equal freedom and abundance for all, after the acceptance and
integration of objective realities.

\subsection{The political psychology of prohibiting radical right-wing
thinkers}\label{the-political-psychology-of-prohibiting-radical-right-wing-thinkers}

\emph{The real motivation of respectable progressivism is managing
guilty conscience and conserving bourgeois privileges; how to theorize
the Cathedral from the revolutionary left; understanding the suicide
bombing of anti-intellectual intellectuals.}

Why do so many respectable, intellectually oriented people (academics,
artists, and activists) wish to exclude independent right-wing
intellectual work on moral grounds? I have in mind several recent
campaigns to ensure some right-wing writer/artist, not be allowed to
speak or perform in some venue. I am going to make a general theoretical
point about a widespread social phenomenon, but I should also be honest
about the specific individuals and institutions I see as motivating case
studies. I have in mind the campaign to
\href{https://shutdownld50.tumblr.com/}{shut down the LD50 gallery}, and
the decision of The New Centre for Research \& Practice to
\href{https://www.facebook.com/thenewcentre/posts/644026572465531}{remove
Nick Land from its roster of lecturers}. These campaigns and
institutional decisions, and so many public pronouncements by other
``progressive'' actors, present themselves as efforts to protect the
public sphere from violent or harmful effects, but it's increasingly
impossible to believe that this is the real motivation. There is a
widespread fallacy, what we might call the ad hominem fallacy fallacy,
that it's unreasonable to question someone's motivations. It may be
unreasonable to dismiss someone's arguments by impugning their
motivations, but it's very reasonable to theorize someone's motivations
as on object of interest in its own right---\emph{especially} when the
stated motivations are increasingly belied by the effects they
repeatedly produce.

I want to offer a concrete, informal theoretical account of what
institutional intellectuals are really doing when they pull-up their
draw-bridges to independent right-wing intellectuals such as Nick Land.
Very simply, they are imposing a \emph{cordon sanitaire} that is
instrumentally necessary to the continuation of their unjustified
intellectual privileges in the institutional order. I am increasingly
convinced there is simply no other public function to this political
repetition compulsion. The reason this is important, from the left, is
that this \emph{cordon sanitaire} is straightforwardly a mechanism to
conserve the status quo, everything progressives pretend to be
interested in overthrowing. This is why neo-reactionary intellectuals
speak of the status quo political order as dominated by a
left-progressive ``Cathedral,'' and this post will provide one example
of why I think they are correct to do so. In this way, genuine left
revolutionaries will sooner than later learn to take heed from some of
these right-wing diagnoses about what is rotten in left culture. If you
believe in radical equality and collective liberation from unjust
institutions, from the left, you have to be able to see how your
compulsions might be a function of this Cathedral more than you care to
admit. As someone who currently possesses bourgeois intellectual
privileges, and as someone who believes in equality (i.e., I know I am
not ultimately superior to many others who lack my platform), I wish to
be perfectly clear about how these mechanisms function, as someone who
bears witness to them introspectively. If it is still possible to be a
genuinely radical intellectual committed to collective liberation, this
is the least it will require.

First, it seems to be a fact that the genuinely intellectual wings of
the alt-right or neo-reaction (NRx) or whatever you want to call it, are
probably too intelligent and sophisticated for bourgeois intellectual
workers to engage with, let alone compete with. The reason I know this
is because I have only been able to really explore this world with the
privilege of my sabbatical; bourgeois intellectual workers typically
just don't have the time to read a bunch of long essays on the internet.
So if those essays are actually pretty smart and a legitimate challenge
to your institutional authority as a credentialed intellectual---you are
functionally required to close ranks, if only with a silent agreement to
not engage. As an academic political scientist, I have at least average
comfort with the history of political thought, yet when I really peruse
all the independent NRx intellectuals, if I'm being honest I'd have to
admit that I would need to go back to the books to really grok and
engage what some of them are trying to say. I am on research leave and I
don't even have the time (or interest) in really doing that as deeply as
would be required to engage all of it meaningfully. This is how I can
very confidently call bullshit on any currently full-time bourgeois
symbol-manipulator who pretends to know with any confidence the alleged
uselessness or harmfulness of the NRx intellectual ecology.

Now, as soon as anyone from this non-institutional world produces
effects within the institutional orbit, it is actually a really serious
survival reflex for all institutionally privileged intellectuals to play
the morality card (``no platform!''). If all these strange, outside
autodidacts are actually smart and independently producing high-level
intellectual content you don't have the time to even understand, let
alone defeat or otherwise control, this is an existential threat to your
entire livelihood. Because all of your personal identity, your status,
and your salary, is based directly on your credentialed, legitimated
membership card giving your writings and pontifications an officially
sanctioned power and authority. If that door is opened even a crack by
non-credentialed outsiders, the whole jig is up for the respectable
bourgeois monopoly on the official intellectual organs of society.

An interesting question is, because respectable intellectuals are often
pretty smart and capable, why are they so fearful of outside
intellectual projects, even if they are as evil as some fear? They are
smart and capable intellectuals, so you'd think they would embrace some
interesting challenge as an opportunity for productive contestation. Why
don't they? Well, here's where the reality gets ugly. The reason
respectable intellectuals so instinctively close ranks around the moral
exclusion of NRx intellectuals is that currently working, respectable
intellectuals privately know that the intellectual compromises they have
made to secure their respectability and careers has rendered most of
their life's work sadly and vulnerably low-quality.

To convince status-quo cultural money dispensers to give you a grant,
for instance, any currently ``successful'' academic or artist has to so
extensively pepper their proposal with patently stupid words and notions
that knowingly make the final result a sad, contorted piece of work 80\%
of which is bent to the flattery of our overlords. But we falsely
rationalize this contortion as ``mature discipline'' which we then
rationalize to be the warrant for our privileged status as legitimate
intellectuals. This is just good old fashioned conservatism, the
standard psychology of bourgeois hypocrisy that is the molecular basis
for the stability of a capitalist society organized around unjust and
unequal exploitation.

Because we know deep down inside that our life's work is only half of
what it could have been had we the courage to not ask for permission, if
there ever arise people who are doing high-level intellectual work on
the outside, exactly as they wish to without anyone's permission or
money, then not only are we naturally resentful, but we secretly know
that at least some of these outsiders are likely doing more interesting,
more valuable, more radically incisive work than we are, because we
secretly know that we earn our salary by agreeing to only say half of
what we could.

(Aside: The only reason I'm psychologically and sociologically capable
of writing this is precisely because through the internet I am shifting
a large share of my eggs into the basket of the outside, instead of
writing grant bids that actively make me less interesting and less
honest; and I am young enough in my academic career that my sunk costs
and interpersonal investments in academic networks are sufficiently low
that I can admit these realities without too much pain.)

Because the entire social value and self-esteem of respectable
intellectuals is premised on this \emph{guilty conscience}, moral
opposition is the most immediately natural and useful symbolic flagpole
around which a \emph{cordon sanitaire} will rapidly and spontaneously be
constructed and defended. In this natural and obvious urgency, bad faith
will occasionally out itself in slips of the tongue whereby academics,
such as a one David Golumbia,
\href{https://twitter.com/dgolumbia/status/848538857331191809}{proudly
and explicitly reveal their commitment to anti-intellectual tactics.}
You might think that rallying around such a patently unintellectual
position typically associated with bourgeois-conservative simpletons
would itself be avoided as an existential threat just as well, for it
loudly publicizes a scandalous dearth of the very intellectual firepower
on which their status and salaries is supposed to depend. That
respectable bourgeois intellectual culture would seek to stave off one
existential threat through the symbolic equivalent of a kamikaze gambit
reflects a secret awareness of its own doomed nature.

\subsection{The affective politics of keeping it
real}\label{the-affective-politics-of-keeping-it-real}

\emph{Why real nobodies become more powerful than repressed somebodies
(the internet epoch has hardly begun).}

It once made sense for academics to bite their tongue in exchange for
the influence they could gain by conforming to the dominant language.
For a while, this was arguably rational and defensible---perhaps even a
game-theoretic necessity for anyone sincerely interested in cultivating
a genuinely public and political intellectual project. While it's
obvious the internet has changed the game, old stereotypes die hard and
continue to constrain human potential well after their objective basis
has disappeared. In particular, the contemporary stereotype of the
public intellectual as a self-possessed professional who regularly
appears in ``the media'' to speak on public affairs in the royal
language, is a contingent product of the postwar rise of mass
broadcasting (one-to-many) media. In much of the postwar period, the
classic ``mass media''---newspapers, radio, television---had extremely
large, mass audiences and where characterized by high costs of entry.
This technical and economic environment offered huge rewards for
speaking the dominant language within the parameters of respectable
opinion. It was probably with cable television that a centrifugal
tendency began the processes of fragmentation, polarization, and
decentralization that would eventually bring us to where we are today.

Today, there is no longer any mass audience to speak to through dominant
channels, overwhelming majorities do not trust mass media, and even the
cognitively fragmented semi-mass audiences that remain will only listen
to what they already think. Not to mention the masses probably have less
power today than anytime in the twentieth century, so why bother even
trying to speak to the masses? As a young academic, if I play by the
rules for the next 10 years so that I might be respected by influential
academics or gain access to regularly speaking on BBC or something like
that, I would have sacrificed all of my creative energy for quite nearly
nothing. As far as I can tell, today, the idea of biding your time as a
young and respectable intellectual, to one day earn a platform of
political significance, appears finally and fully obsolete. In one
sense, this is already obvious to the millions who long ago stopped
following mainstream media and long ago lost all respect for academic
credentials; but in another sense, an overwhelming number of human
beings continue to think, speak, and behave as if we are still operating
in this old world, as if there is some reason to not say everything one
feels like saying, as if there is some social or political or economic
reward that will come toward the end of a respectable career of
professional self-restraint. It's easy for autodidacts and natural
outsiders to say, ``Duh, we told you so,'' but this in no way
comprehends or solves the really striking and politically significant
puzzle that an extraordinary degree of human power remains
\emph{voluntarily repressed} for rewards and punishments that no longer
exist.

Just as the self-restrained professional intellectual is shaped by the
rewards of a media environment long dead, so too are they shaped by
punishments which are little more than paranoid fears. Many academics
and professionals believe that for the sake of their careers they must
exercise the utmost discretion in what they put online, and they
confidently tell young people to exercise the same discretion for the
sake of their own futures. But the reality is almost the exact opposite.
First of all, with some important exceptions of course, nobody gives a
shit about what you put on the internet. Nobody with any power over you
has the time to follow, and the few that do won't care enough about you
to follow or dig very much. In my now slightly above-average history of
recklessly posting to the internet, before and after getting a
competitive professional job, the worst that has ever happened is that
nobody cares (and that's most of the time). But the best that has
happened, here and there, is that a lot of people care and appreciate it
and new friends are made and all kinds of new paths appear, individually
and collectively.

The self-restraining, strategic professional intellectual is not only
operating on incorrect beliefs but beliefs which are almost exactly
inverse to the truth: today, playing by rules of respectability is
perhaps the straightest path to unemployment and impotent resentment,
while simply \emph{cultivating the capacity to say or do something real}
(by definition prohibited by respectability), is a necessary (and
sometimes even sufficient) condition for being genuinely valued by
anyone, anywhere. Obviously if you have certain dimensions of poor
character (i.e.~you're a racist or something) then reckless posting to
the internet will likely, and perhaps rightly, lead to many negative
consequences. But if you're a basically decent person who just wants to
push a little harder on what you really think, what you really feel,
your experiences or your interests, or even just fuck around, the
conventional wisdom still drastically overestimates the punishments and
underestimates the rewards of doing so.

If these comments feel to you outdated because you think all of this
already happened years ago with the initial rise of the internet, I
would say you underestimate the quantity of human beings (and the
qualitative intensities they could produce), who have yet to fully
update their beliefs and behaviors around these matters. In some sense,
the United States only just now, in 2016, elected its first President of
the internet age. The fact that millions of people are genuinely
perplexed and horrified by what is happening in this regard, is an index
of how little the internet's rewiring of power circuits has actually
been integrated in the perceptions, beliefs, and behaviors of most
people.

If political theory will ever be a world-historical causal factor in the
radically fragmented and decentralized age of information, I believe it
will at least partially have to consist in living, interpersonal
transmissions of \emph{what works} to produce power within concrete,
available situations. I suppose it is on that belief that I am
reflecting on matters such as the micro-politics of self-expression,
with some personal context as anecdotal data. I believe there exist
objective, micro-political mechanisms whereby \emph{being real}
generates \emph{real power}; that many people under-estimate or mistrust
the objective reality of this mechanism; that many people live under
compliant resentment because of incorrect beliefs about how the
macro-social institutional environment will respond to their
idiosyncratic deviations.

If I speak about myself at all, it's not because I'm special or great
but precisely because I am nothing, nobody, yet it has always been in
learning how to become nobody that I seem to unlock whatever few real
powers now possess me. I'm a young nobody academic or writer or
whatever, I'm not famous or influential in academia or even on the
internet, but I have been able to cultivate and maintain an energetic,
autonomous, creative intellectual life that feels to me on the right
track intellectually and politically, by what I think is a more accurate
but still uncommon map of our immediate strategic environment, combined
with periodically forcing myself to push outward on what feels to be in
some way not allowed.

I have almost come to see this as a methodological principle---wherever
you genuinely think or observe or desire to do something that you would
be vaguely punished socially for saying/doing at a dinner party, then
saying/doing that on the internet almost always produces immanent power
and its by-product joy; it almost always clears debilitating noise and
advances one of the threads that make-up the continuity of one's life.
The entire life orientation that believes facing toward power is the
image of intellectual and political sophistication---this, in my view,
is the spitting image of mystified contemporary complicity. If I am
trying to share how I think all of this works, and how I have
experienced it, it is because I dream of what would happen if thousands
of highly capable intellects currently toiling under institutional
respectability suddenly realized they have no reason to self-censor and
everything to gain from simply disarming their objectively miscalibrated
expression calculators.

\subsection{Feminism and the problem of supertoxic
masculinity}\label{feminism-and-the-problem-of-supertoxic-masculinity}

\emph{A serious feminist challenge is what to do with hyper-dominant
males who are not domesticated by any amount of moral or legal
constraint; in fact, excessive social constraints on average males may
help explain why, today, supertoxic masculinity is now a fast-track to
the heights of political power.}

I just watched \href{http://www.imdb.com/title/tt6071534/}{a documentary
film about John McAfee,} creator of the famous McAfee Antivirus you will
remember from every PC in the 1990s. I didn't know anything about this
man before watching this film. I want to make a point that requires me
to give you a brief summary of the story, so here is a plot-spoiling
recapitulation. Basically, after he gets rich off McAfee Antivirus,
McAfee has a couple of failed business ventures before proceeding
through a brazenly aggressive, daring, manipulative, controlling,
arrogant, violent, and ultimately murderous course of affairs.\footnote{Of
  course, I am assuming the film's narrative is to be trusted. I haven't
  fact-checked anything. Whether the film is perfectly accurate or fair
  is probably not crucial for the larger point I will make here.} After
a few years as a yoga guru preaching peace and wellness from a retreat
center he funded, there's some indication that he becomes disillusioned
with his efforts toward egalitarian community (he suggests something to
the effect that others were taking advantage of him, but this is not
examined deeply). So he buys a house in Belize, hires an idealistic
biologist from the US to run an alternative medicine laboratory,
recruits the toughest gangsters he can find to build an in-house private
security force, donates to local police equipment worth millions of
dollars, and effectively purchases several poor, local women as
long-term girlfriends. When he had the time to also get two gnarly
tribal tattoos was unclear to me.

Just to round out the psychological and behavioral profile here, note
that he rarely, if ever, had sex with his girlfriends; he rather liked
to defecate in their mouths while lying in a hammock. When the biologist
expresses concern about their business relationship, he drugs and rapes
her \emph{that evening,} according to the biologist's testimony in the
film. McAfee's vicious guard dogs roamed freely on the public beach
around his house, so a neighbor poisoned the dogs. Then, the film
suggests, McAfee promptly hired a man to kill the neighbor. This murder
allegation becomes global news, and McAfee embarks on an international
fugitive escape adventure. He gets into Guatemala, where he avoids
extradition back to Belize by faking a heart attack, and thereby
engineering his deportation back to the United States. He then promptly
runs for President in the Libertarian Party, where he comes in second
place.

Now, it is striking enough that the winner of the 2016 Presidential
election is an icon of ignoring feminist ethical expectations---at a
time when feminist expectations are more culturally ascendant than ever.
But perhaps that was a fluke. The McAfee story is profound because it
shows in stunning, horrifying detail how the hyper-masculine drive to
dominate really works in contemporary culture: when cranked sufficiently
high, it rapidly and easily trounces any quantity of moral outrage
and/or legal constraints, in a direct line toward the zenith of the
global dominance hierarchy.

Moderate misogyny can get you exiled from contemporary public culture,
often for good reason, but hyper-misogyny in an intelligent and driven
male appears to give you sovereignty \emph{over} public culture. It
seems to me that, if feminism today has one genuinely catastrophic
problem to be rightfully alarmist about, it might just be the small
number of males who will not be domesticated through social-moral
pressure.

First, a premise of my argument is that SJW culture is genuinely quite
effective at minimizing the nastier masculine edges of large numbers of
men, because most men are decent people who want to be liked and
approved by most others. This is not an empirical article so I won't go
into it, but if you doubt there's been a general cultural pacification
of male aggression just watch a random film from the 1950s and then
watch a random film at your local cinema. Anyway, people on the left and
right disagree about what to call this trend, but its existence is
attested by all. Feminists see this as men learning to be less violent
and oppressive, and feminists celebrate women's long-term positive
effect on the civilizing of violent patriarchies; others see this as a
kind of female totalitarianism and evidence of civilizational decline.
But the fact that feminist cultural politics have exerted notable and
widespread effects of generally reducing the expression of masculine
aggression in public culture seems hard to dispute.

The hypothesis I would like to advance is that this social domestication
of masculine tendencies has made our society more vulnerable to the rare
cases of men who escape the filter of social opprobrium. The life of
John McAfee is a case study of this problem.

Why would the social pacification of once popular, moderate masculinity
empower more virulent forms of violent masculinity? Many lefties think
that pacifying the larger mass of men will shift the whole distribution
of male behavior, lowering the ceiling of how bad the worst men may
become. I would say this is the dominant mental model of most SJWs,
because it's the basic picture that comes out of liberal arts education
today (that our images of the world shape what we do in the world, hence
the emphasis on media and ``representations'').

The problem is that when the baseline of masculine dominance expression
is held below it's organic tendency, defined simply as what men would do
in the absence of cultural campaigns to defang it, this increases the
potential payoff to those who dare exercise it, as there are more
resources to dominate precisely to the degree that other men are not
contesting them. Not only does it increase the rewards available, it
decreases the risk of competing for them, as the chance of being
defeated by an equally aggressive male, or even just the chance of
encountering costly competition at all, is lower than it would be in a
world of much but minor, local masculine excess. We might also adduce a
``rusty monitor'' effect: Through the domestication of men over time,
most people become blissfully forgetful about what genuinely dangerous
men are capable of, decreasing the probability or the speed with which
domesticated males might awake from their slumber.

Another reason the over-domestication of moderate masculinity is
dangerous is that it makes it too easy for ethically lax ``bad
characters'' to win all of the large number of local hierarchies that
would typically have the function of imposing humility and modesty on
cocky boys coming of age. If you're a highly intelligent, confident, and
driven young man, the complex difficulty of having to navigate multiple
distinct local hierarchies (among other highly driven males themselves
sometimes prone to dangerous excess) from a young age, teaches you very
quickly that you cannot ever be the best at everything. And that if you
cut corners anti-socially you will be destroyed by other males invested
in the maintenance of sociality. Examples of local hierarchies are
sports competition, dating, ethical honor or ``character'' in the
neighborhood or religious community, or even just fleeting micro-social
competition such as battles of wits in social gatherings. All of these
things will function as negative feedback mechanisms tempering genuinely
dangerous anti-social ambitions in young boys coming of age, but only if
the other males are equally able and willing to play all of these games
to the best of their abilities.

If you're overzealous or immodest or you cheat or you ignore your
standing in one local hierarchy to dominate another---all of these
things tend to get constrained by other males of equal will and ability,
who are also sometimes dangerous and who have an interest in knocking
all wiley characters down a few notches. What's happened in recent
decades is that a non-trivial portion of the West's most intelligent and
ambitious males pursue cultural careers predicated very specifically on
the strategic under-display of their will to power. Take someone like
the Canadian Prime Minister Justin Trudeau---he's the leading politician
of a whole country, so nobody can deny that this is a man with a
substantial will to rise to the top through a whole series of
competitive filters. But he is one of the best examples of how, today,
the path to power for all ``decent men'' consists in a deeply deceptive
competition to appear maximally unthreatening. One reason you get the
John McAfee's of the world is because they went to high school with the
Justin Trudeau's of the world. In all of the little, local hierarchies
they encountered throughout life, people like John McAfee and Donald
Trump learned that they could be as anti-socially ambitious as they
pleased and no other intelligent and able men would check them (because
those men were opting for the cultural capital that accrues to being
feminist). A serious challenge for feminism is to see that someone like
Justin Trudeau is seriously complicit in the production of the McAfees
and Trumps of the world. And if your a cheer-leader for the former,
you're an objective supporter and producer of the latter.

I also think that people like McAfee and Trump learn early in life that
if you are ostracized from social groups for exceeding moral
expectations, then you can just channel your anti-social intelligence to
making money all the more efficiently. That is, another key problem is
that in secular, advanced capitalist countries such as the U.S., if you
are smart and driven enough it is a feasible life path to accept
absolute social exile by converting all of your energy into economic
capital accumulation, and then build up a new social cosmos for
yourself. The interesting thing is to see that this is really only
psychologically and materially feasible in a very late stage of advanced
western capitalism where non-economic criteria of value have all but
disappeared. Whereas above we saw one reason for the emergence of the
McAfee's and Trump's of the world is that there wasn't enough local
masculine aggression to check them throughout their life, here we note
the specific problem that secular society lacks any effective
adjudicator of human character other than economic prowess. In this
particular dimension we see that the contemporary correlation of
anti-capitalism and secularism/atheism is ultimately an untenable loop,
because you never have an effective basis for anti-capitalist cultural
change if you cannot submit to the possibility that values come from a
place higher than practical reality. Of course people pretend they value
other criteria, but those criteria don't \emph{operate} in the selection
of who ultimately wins attention, esteem, and power in society as a
whole. There was no person, and no entity, in the entire life of these
men who could credibly convey that there exist things in life more
powerful than money, for the simple reason that hardly anyone believes
this anymore. And so the most toxically ambitious males become the very
first to realize that one can very well quit the entire game of
socio-moral respectability and shoot to the top of everything via
radically unreflective capital accumulation.

Another reason why the constraining of moderate masculine toxicity may
increase the power of supertoxic masculinity is that males may become
more pathologically power hungry from lacking opportunities for healthy
satiation. Once upon a time (for better or worse), masculine prowess
promised a fair number of immediate satisfactions. The best football
players received the genuine interest of the most desired girls in high
school, say. But even from my own observations growing up, it was easy
to see that as my cohort aged from about 10 years old up toward about 17
years old, conventionally masculine prowess became less and less
effective at winning immediate social rewards. By the end of high
school, the most desired girls were more interested in---I kid you
not---a nationally competitive business role-playing team. What this
suggests to me is that, aside from perhaps an early bump at the very
beginning of adolescence, dominance hierarchies rapidly stop rewarding
conventional masculine expressions of dominance behavior in favor of the
capacity to elegantly dissimulate dominance behavior. Today all of the
basic evolutionary machinery of mating and dominance competition remains
in full operation, but it's mind-bogglingly confusing because
increasingly females select for males who can most creatively and
effectively hide their power. What this means is that precisely the most
over-flowingly aggressive males may be less and less likely to receive
the basic, small doses of love and esteem that every human being
requires, in their early socialization experiences. Combined with the
previous point about the ultimate power of money, it's easy to see how
and why the feminist inversion of which males get selected by females
(defining dominance as the dissimulation of dominance), has the direct
consequence of leaving the most irrepressibly narcissistic and
power-hungry males to seek unbridled social domination via capital, as a
basic requirement for psychological self-maintenance.

John McAfee and Donald Trump are the types of whom it can be said,
literally, that they are capable of making the entire world conform to
their whims. They can do this repeatedly and sustainably, even when a
large number of interested opponents see what they are doing, even when
it publicized to the moral outrage of the entire respectable,
cosmopolitan world. What is genuinely frightening and dangerous about
powerful males is precisely that their power is real, i.e.~absolutely
impervious to the wishes, interests, and indignant words of less
powerful people.

It seems to me that, broadly, there are two possible ways to dealing
with this problem. The method popular activist culture has adopted is to
work toward a state of zero dominance expression in all possible local
and global hierarchies, with this leading to a substantially higher risk
of psychopathic males going straight to the top of the megamachine, for
all of the reasons I've laid out. Now, to be fair, I see one way you
might still find this method preferable: if you believe that
psychopathic male drives for dominance could possibly be socialized out
of our biology altogether in some kind of long-term evolutionary
engineering process. If that's your model, then I suppose you could
defend the now popular approach as a risky but radical plan to eliminate
violence forever, or something like that. Personally, I find that hard
to believe, but that would require a different essay. In the meantime, I
suppose we all have to make our wagers as we see fit.

Of course, the second solution is simply to permit or even encourage
small amounts of masculine dominance behavior in a large number of local
hierarchies (with some margin greater than zero for dangerous excesses),
leading to a low likelihood of psychopathic males rising to the top of
the megamachine.

A final point about the role of higher education in all of this. In a
contemporary liberal arts education, the primary educational experience
is coming to feel the power of words. This is a real and important
insight because in modern societies the symbolic order exerts
extraordinary if diffuse effects, and I benefitted from gaining this
kind of awareness in my own liberal arts education. This feeling is also
exciting and empowering because we all have the capacity to produce
words. But for this reason---combined with the fact that direct violence
in wealthy Western societies is atypically low in long-run historical
perspective---a very large number of well-meaning lefty folks today have
genuinely forgotten that there exist forces more powerful than words. We
have forgotten that the whole, horrifying, tragic, and very real problem
of power is precisely that those who have enough of it may ultimately do
exactly what they please. Many lefties today seem to be living on the
genuine belief that enough people, saying enough words, is a viable
method for constraining anything whatsoever. It's not.

The McAfee documentary is an extraordinary lesson of how no amount of
moralizing can solve the fact that unequal distributions of raw human
power exist across society; no amount of ``awareness'' or
information-sharing or even law-making will ever be able to stop the
will to power wherever it sneaks through the cracks of social
inhibition. One of morality's dirtiest and most harmful little secrets
is that it only constrains power where power is already weak for other
reasons. Contemporary SJW-styled feminism will make the large mass of
beta bro's marginally more polite. It may, for short- to medium-term
intervals, suppress the brutality of alpha types who may indeed be prone
to some abusive behaviors. But it will also ensure that wherever the
male will to dominance arises in it purest form, it will wreak more
havoc, more rapidly, more unpredictably, more completely, and at a
higher socio-political level than it ever could have without feminist
``moral progress.''


\end{document}
